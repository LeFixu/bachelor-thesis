\newglossaryentry{watson}{
    name=Watson,
    description={Schweizer Nachrichtenportal welches seit dem 22. Januar 2014 online ist.}
}

\newglossaryentry{tfidf}{
    name=TFIDF,
    description={Term Frequency Inverse Document Frequency. Eine Technik zum vorprozessieren von Text für \ashort{ml}}
}

\newglossaryentry{naive-bayes}{
    name=Naive Bayes,
    description={Ein statistisches \ashort{ml} Modell.}
}

\newglossaryentry{document-store}{
    name=Document Store,
    description={Eine Dokumentbasierte Datenbank wie MongoDB. Gehört zur Klasse der No-SQL Datenbanken. }
}

\newglossaryentry{beautifulsoup}{
    name=BeautifulSoup,
    description={Eine Schnittstelle zum programmatischen Steuern von Browsern \cite{beautifulsoup}. Wird meist für UI Tests verwendet.}
}

\newglossaryentry{scrapy}{
    name=Scrapy,
    description={Eine Python-basierte Standardlösung zum Crawlen von Webseiten.}
}

\newglossaryentry{refactorn}{
    name=Refactorn,
    description={Programmcode überarbeiten}
}

\newglossaryentry{geschlechterbias}{
    name=Geschlechterbias,
    description={Wahrnehmungsverzerrungseffekte in Bezug auf Gender, auch Genderbias genannten.}
}

\newglossaryentry{collection}{
    name=Collection,
    description={Eine Sammlung von Dokumenten in MongoDB. Vergleichbar mit einer Tabelle in einer relationalen Datenbank.}
}

\newglossaryentry{genderbias}{
    name=Gender Bias,
    description={\gl{geschlechterbias}}
}

\newglossaryentry{gendergap}{
    name=Gender Gap,
    description={Unterschied zwischen Geschlechtern}
}

\newglossaryentry{paygap}{
    name=Pay Gap,
    description={Lohnunterschied (zwischen den Geschlechtern)}
}

\newglossaryentry{datagap}{
    name=Data Gap,
    description={Unterschied in der Repräsentation der Geschlechter in Datensets}
}

\newglossaryentry{sex}{
    name=Sex,
    description={Biologisches Geschlecht}
}

\newglossaryentry{gender}{
    name=Gender,
    description={Das Geschlecht, als das sich eine Person identfiziert. Ist nicht zwangsläufig gleich wie das biologische Geschlecht (\gl{sex})}
}

\newglossaryentry{ringier}{
    name=Ringier,
    description={Ein grosser Medienkonzern der Schweiz, zu dem auch Blick gehört.}
}

\newglossaryentry{glass-ceiling}{
    name=Glass Ceiling,
    description={Der Begriff "gläserne Decke" bezieht sich auf eine unsichtbare Barriere oder Einschränkung, die Frauen daran hindert, in höhere Positionen am Arbeitsplatz aufzusteigen, insbesondere in Macht- und Führungspositionen. Der Begriff wird in der Geschlechterforschung und der Soziologie verwendet, um die systemischen und oft subtilen Formen der Diskriminierung und Ungleichheit hervorzuheben, mit denen Frauen im beruflichen Umfeld konfrontiert sind. Die gläserne Decke steht für die tief verwurzelten geschlechtsspezifischen Vorurteile und strukturellen Hindernisse, die den beruflichen Aufstieg von Frauen behindern und die Ungleichheiten zwischen den Geschlechtern in der Arbeitswelt verstärken.}
}

\newglossaryentry{parsetree}{
    name=Syntaxbaum,
    description={Hierarchische Darstellung der Zergliederung eines Textes.}
}

\newglossaryentry{dependencytree}{
    name=Abhängigkeitsbaum,
    description={Ein Abhängigkeitsbaum für einen Satz ist ein gerichteter azyklischer Graph mit Wörtern als Knoten und Beziehungen als Kanten.}
}

\newglossaryentry{quote-syntactic}{
    name=Syntaktisches Zitat,
    description={Ein Zitat, das über Subjekt, einleitendes Verb und Aussage verfügt}
}

\newglossaryentry{quote-floating}{
    name=Schwimmendes Zitat,
    description={Ein Zitat ohne Subjekt. Im \ashort{ggt} \enquote{Floating Quote} genannt. Meist in der direkten Rede und in der Nähe eines \gl{quote-syntactic}. \enquote{Das Wetter ist schön} hat Mark laut gesagt. \enquote{Obwohl es etwas Wolken hat}}
}

\newglossaryentry{quote-direct}{
    name=Direktes Zitat,
    description={Ein Zitat in der direkten Rede: \enquote{Frau Keller-Suter jauchzte \enquote{Wir kriegen neue Kampfjets!}}}
}

\newglossaryentry{quote-indirect}{
    name=Indirektes Zitat,
    description={Ein Zitat in der indirekten Rede: \enquote{Frau Keller-Suter jauchzte dass wir neue Kampfjets kriegen.}}
}

\newglossaryentry{dom}{
    name=DOM,
    description={Steht für \along{dom-acronym}. Der Objekte Baum einer Webseite zur Laufzeit.}
}

\newglossaryentry{html}{
    name=HTML,
    description={Steht für \along{html-acronym}. Das erste File, das von einer Webseite geladen wird. Grundbaustein für den Aufbau einer Webseite.}
}

\newglossaryentry{bs4}{
    name=BS4,
    description={Steht für \along{bs4-acronym}. Eine Parsing Library für \gl{html} oder \ashort{xml}}
}

\newglossaryentry{politeness}{
    name=Politeness,
    description={Englisch für \enquote{Höflichkeit}. Beschreibt Einstellungen von Web Crawlern, die eine Überlastung des Webservers verhindern sollen.}
}

\newglossaryentry{seed-url}{
    name=Seed URL,
    description={Die Start-\ashort{url} der Webseite, die gescraped werden soll. Seed ist Englisch für Samen. Die Seed URL ist also der Samen des Crawling-Baums.}
}

\newglossaryentry{selenium}{
    name=Selenium,
    description={Selenium ist eine Schnittstelle für Applikationen um Web Browser zu steuern. 
    Selenium wird häufig für automatisierte Browser Tests und Web Crawling verwendet.}
}

\newglossaryentry{equalvoice}{
    name=EqualVoice,
    description={Eine Initiative vom Ringier Verlang um die Präsenz von Frauen auf ihren Plattformen zu messen \cite{ringier-equalvoice}.}
}

\newglossaryentry{glueckspost}{
    name=Glückspost,
    description={Frauenzeitschrift für Prominenz und Unterhaltung \cite{glueckspost}.}
}

\newglossaryentry{bilanz}{
    name=Bilanz,
    description={Unterseite der Webseite handelsblatt.ch mit dem Titel Bilanz | Das Schweizer Wirtschaftsmagazin | BILANZ \cite{bilanz}.}
}
\newglossaryentry{kanban}{
    name=Kanban,
    description={Kanban ist eine Methode um Tasks zu verwalten. Mit Hilfe eines Kanban Boards wird der Stand der Tasks visualisiert.}
}
\newglossaryentry{pep8}{
    name=PEP8,
    description={Ein Python Code Standard \cite{pep8}}
}
\newglossaryentry{mypy}{
    name=MyPy,
    description={Ein Linting Tool für Python Programme, welches das Static Typing überprüft}
}
\newglossaryentry{pylint}{
    name=PyLint,
    description={Ein Linting Tool für Python Programme, das den Code Stil überprüft}
}

\newglossaryentry{gender-bias-distribution}{
    name=Gender Bias in der Verteilung,
    description={Der Unterschied in der Verteilung der Geschlechter (in einem Datenset)}
}
\newglossaryentry{gender-bias-content}{
    name=Gender Bias im Inhalt,
    description={Bias in der Bedeutung der Beschreibungen unterschiedlicher Geschlechter.}
}
\newglossaryentry{gender-bias-wording}{
    name=Gender Bias in der Wortwahl,
    description={Bias in der Wortwahl mit der Personen unterschiedlicher Geschlechter beschrieben werden.}
}
\newglossaryentry{possessive-noun}{
    name=Possessive Noun,
    description={Englisch für ein Nomen, dass Besitz oder eine Richtung beschreibt. Beispiel: \enquote{It's Maria's dog}.
    \enquote{Maria's} is a possessive noun}
}
\newglossaryentry{attribute-word}{
    name=Attribute Word,
    description={Englisch für ein Wort, dass einem Subjekt eine oder mehrere Eigenschaften zuschreibt. Beispiel: \enquote{Politikerin}}
}
\newglossaryentry{vader}{
    name=VADER,
    description={Ein tool für Sentiment Analysis}
}
\newglossaryentry{lemming}{
    name=Lemmatisierung,
    description={Lemmatisierung ist die Reduktion eines Wortes auf die Grundform (das Lemma)}
}
\newglossaryentry{precision}{
    name=Precision,
    description={Der Bruchteil der gefundenen relevanten Daten über alle relevanten Daten im Datensatz}
}
\newglossaryentry{recall}{
    name=Recall,
    description={Der Bruchteil der gefundenen relevanten Daten über alle als relevant eingestuften Daten}
}
\newglossaryentry{f1-score}{
    name=F1 Score,
    description={Das Harmonische Mittel aus \gl{precision} und \gl{recall}}
}
\newglossaryentry{cr}{
    name=Coreference Resolution,
    description={Ein Algorithmus zum Clustern von zusammengehörigen Wörtern, die von einem \gl{spacy} Parser analysiert wurden.}
}
\newglossaryentry{mathilda-effect}{
    name=Mathilda Effekt,
    description={Beschreibt die systematisch kleinere Anerkennung von Wissenschaftlerinnen verglichen mit Wissenschaftlern}
}
\newglossaryentry{kbp}{
    name=Knowledge Base Population,
    description={\gl{kbp-acronym} bezieht sich auf die automatisierte Extraktion von strukturierten Informationen aus unstrukturierten Texten, um eine Wissensdatenbank aufzubauen oder zu erweitern. }
}

\newglossaryentry{dependency-parsing}{
    name=Dependency Parsing,
    description={Eine \gl{nlp} Technik, die die Abhängigkeiten von Wörtern innerhalb von Sätzen bestimmen kann.}
}

\newglossaryentry{spacy}{
    name=Spacy,
    description={Eine \gl{nlp} Library \footnote{https://spacy.io/}, die \gl{ner}, \gl{pos} und \gl{dependency-parsing} beinhaltat.}
}

\newglossaryentry{spacy-parser}{
    name=Spacy Parser,
    description={Ein Teil der \gl{spacy} Komponente.}
}

\newglossaryentry{cosine-similarity}{
    name=Cosine Similarity,
    description={
Ein Mass für die Ähnlichkeit zwischen zwei Sätzen oder Vektoren, 
das den Kosinus des Winkels zwischen ihnen berechnet. 
Es basiert auf dem Vector-Space und Bag Of Words Modell.}
}

\newglossaryentry{levenshtein-similarity}{
    name=Levenshtein Ähnlichkeit,
    description={
Verwendet die Levenshtein Distanz als grundlage,
die beschreibt die Anzahl Veränderungen um von einem 
Text A zu einem Text B zu gelangen. 
Die Ähnlichkeit ist eine Normalisierung dieses Verfahrens, 
welche die Distanz auf den Raum von null bis eins projiziiert.}
}