\subsection{Academic medicine’s glass ceiling: Author’s gender in top three medical research journals impacts probability of future publication success}
In der Arbeit \citetitle{academic_medicine_glass_ceiling} \cite{academic_medicine_glass_ceiling} haben \citeauthor{academic_medicine_glass_ceiling}
die drei unten aufgelisteten Zeitschriften auf das Geschlecht der Autorinnen und Autoren untersucht.
Diese Zeitschriften haben laut \citeauthor{academic_medicine_glass_ceiling} den höchsten Einfluss in der medizinischen Forschung.
\begin{enumerate}
    \item \gl{nejm}
    \item \gl{jama}
    \item \textsl{Lancet}
\end{enumerate}

Medizinische Fachzeitschriften haben einen erheblichen Einfluss auf die klinische Praxis und die Empfehlungen
von medizinischen Fachgesellschaften. 
In der akademischen Medizin gibt es jedoch nach wie vor geschlechtsspezifische Ungleichheiten, 
insbesondere in Bezug auf die Forschungsproduktivität und den beruflichen Aufstieg \cite{marcotte2021toward,alonso2021gender,chadwick2020gender}. 
Frauen sind seltener in Spitzenzeitschriften vertreten und ihre Arbeiten werden seltener zitiert \cite{benjamens2020gender,kramer2019sex,menzel2019gender}. 
Die Zeitschrift \textsl{Lancet} hat sich zur Untersuchung geschlechtsspezifischer Ungleichheiten verpflichtet \cite{schwalbe2018time}. 
Diese Studie hat gezeigt, dass Frauen seltener als Hauptautorinnen und -autoren, Zweit- oder Seniorautorinnen und -autoren 
in medizinischen Spitzenzeitschriften vertreten sind.

\subsubsection{Materialien und Methoden}
% Geschlecht Bestimmen
Um die primäre Variable der Studie, das Geschlecht, zu ermitteln, haben die Forschenden Informationen wie
Pronomen, Vornamen und Fotos aus Biografien, Lebensläufen und Webseiten extrahiert. 
Wenn das Geschlecht nicht eindeutig festgestellt werden konnte, wurde es auf der Grundlage einer Diskussion im Studienteam 
zugewiesen oder als \enquote{unbekannt} bezeichnet. 
Zusätzlich zu Geschlecht haben die Forschenden auch andere Autorinnen und Autoren- und Publikationsmerkmale, wie akademische Abschlüsse, 
Fachgebiete und Studienergebnisse, extrahiert. 
Die Zuverlässigkeit der webbasierten Datenextraktion, einschliesslich des Geschlechts, haben die Forschenden durch eine hohe  \enquote{inter-rater reliability} bestätigt.

% Berechnung des ursprünglichen Stichprobenumfangs
Die Studie untersuchte geschlechtsspezifische Unterschiede in drei medizinischen Fachzeitschriften mittels eines Chi-Quadrat-Tests. 
Um den Stichprobenumfang zu berechnen, haben die Forschenden Effektgrössen mithilfe des R-Pakets \enquote{pwr} geschätzt. 
Die Forschenden haben 1'033 Publikationen von Erstautorinnen und -autoren verwendet, um die gewünschten geschlechtsspezifischen Unterschiede festzustellen. 
Die Studie konzentriert sich hauptsächlich auf Erstautorinnen und -autoren und die Forschenden haben 20 Artikel pro Jahr und Zeitschrift nach dem 
Zufallsprinzip ausgewählt, was insgesamt 1'080 Veröffentlichungen ergibt. 
Geschlechtsunterschiede bei Zweit-und Letztautorinnen und -autoren sowie bedeutenden Autorinnen und Autoren können aufgrund des Stichprobenumfangs höchstwahrscheinlich 
nicht festgestellt werden.

% Statistische Analysen
Folgende statistischen Methoden haben die Forschenden für die Analyse der Publikationsraten von Frauen und Männern 
in medizinischen Zeitschriften verwendet. 
Die Forschenden haben verallgemeinerte Schätzgleichungsmodelle (GEE) eingesetzt, um Publikationsraten von Frauen 
und Männern innerhalb und zwischen verschiedenen Zeitschriften zu vergleichen. 
Chi-Quadrat-Tests haben die Forschenden verwendet, um Merkmale auf der Ebene von Autorinnen und Autoren zu vergleichen, wie zum Beispiel Fachgebiet, Abschluss, 
Führungsposition und geografische Lage. 
Zeitliche Vergleiche wurden durchgeführt und die Forschenden haben exakte Tests nach Fisher verwendet, um die Mehrfachveröffentlichungsraten 
zwischen Frauen und Männern in verschiedenen Autorinnen und Autorenrollen zu vergleichen. 
Die Ergebnisse haben die Forschenden mit einem p-Wert von $\leq 0.05$ als statistisch signifikant betrachtet. Etwas höhere 
p-Werte von bis zu p $\leq 0.15$ wurden als Trends eingestuft. 
Alle statistischen Analysen wurden mit SAS 9.4 durchgeführt.


\subsubsection{Ergebnisse}
% Ungleichheiten zwischen den Geschlechtern
Die Studie stellt fest, dass Frauen insgesamt weniger häufig als Erst-, Zweit- oder Letztautorinnen auftraten als Männer. 
Insgesamt und für jede der drei führenden medizinischen Forschungszeitschriften wurden geschlechtsspezifische Unterschiede 
bei den Erst-, Zweit- und Letztautorinnen Rollen festgestellt. 
Der Anteil der Frauen als Erstautorinnen lag insgesamt bei 26.82 \%.
Dieses Geschlechtergefälle blieb stabil, im ganzen Zeitraum den untersuchten 17 Jahren. 
Die Autorinnen hatten zudem tendenziell weniger Zitierungen, waren weniger häufig Koautorin von veröffentlichten Artikeln, 
besassen weniger häufig einen Doktortitel und hatten seltener Führungspositionen inne.
Zudem veröffentlichten Frauen weniger klinische Studien als Beobachtungsstudien (p < 0.001), und ihre Projekte konzentrierten 
sich häufiger auf Infektionskrankheiten als die von Männern, deren Projekte sich am häufigsten auf kardiovaskuläre Themen konzentrierten (p < 0.001).
Frauen waren auch seltener als Zweit- und Letztautorinnen vertreten. 
Bei 34.89 \% der Veröffentlichungen in medizinischen Forschungsjournalen waren Frauen Zweitautorinnen. 
Ein Geschlechtergefälle wurde sowohl bei den Zeitschriften insgesamt als auch innerhalb der einzelnen 
Zeitschriften festgestellt: 27.32 \% im \gl{nejm}, 34.12 \% im \textsl{Lancet} und 43.81 \% im \gl{jama}, (alle p < 0.01).
Die Gesamtrate der weiblichen Letztautorinnen lag bei 18.60 \%.
Diese variierte zwischen 15.08 \% im \gl{nejm}, 19.83 \% im \textsl{Lancet} und 20.96 \% im \gl{jama} (alle p < 0.001). 
Die Unterschiede bei den Veröffentlichungsquoten von Frauen waren bei der Rolle der letzten Autorin am dramatischsten.


\subsubsection{Diskussion}
Autorinnen, die in medizinischen Spitzenforschungsjournalen publizieren, sind nach wie vor unterrepräsentiert. 
Trotz Bemühungen, die Sensibilität dafür zu erhöhen. 
Diese Studie zeigt, dass Frauen nicht nur als Erst-, Zweit- und Letztautorinnen in hochrangigen medizinischen Fachzeitschriften 
unterrepräsentiert sind, sondern dass auch erhebliche Unterschiede in der Vertretung von Wissenschaftlerinnen in diesen Zeitschriften bestehen. 
Die Veröffentlichung in diesen wichtigen Zeitschriften gilt als Indikator für den beruflichen Erfolg und das künftige akademische Potenzial 
von Autorinnen und Autoren, was für Beförderungen, Fördermittel und Führungspositionen ausschlaggebend sein kann. 
Die Unterschiede zwischen den Publikationsergebnissen von Männern und Frauen in medizinischen Zeitschriften tragen zur 
Aufrechterhaltung der Geschlechterungleichheit in der gesamten Medizin bei. 
Die Studie zeigt dramatische Unterschiede in der Häufigkeit auf, mit der Frauen als Autorinnen erscheinen,
im Vergleich zu Männern, was Anlass zu Bedenken hinsichtlich der Geschlechterungerechtigkeit gibt.
