\subsection{Cancer in the news: Bias and quality in media reporting of cancer research}

% Generelle Beschreibung + Ziele
Die Studie \citetitle{cancer-in-the-news} von \citeauthor{cancer-in-the-news} \cite{cancer-in-the-news}
untersucht die Medienberichterstattung von ausgewählten grossen Medienhäusern im Angelsächsischen Raum.
Sie zeigt die Verteilung der Berichterstattung über die Art der Studien (Primär- oder Sekundärliteratur) auf,
untersucht die Qualität der entsprechenden Artikel und analysiert den Bias in Bezug auf das Geschlecht und die
Nationalität.
% Motivation ergänzen

% Methoden + Datengrundlage
Als Datengrundlage verwenden die Forschenden je 20 Nachrichtenartikel aus den Online-Nachrichtenportalen von
\textsl{The Guardian} (Edition aus dem Vereinigten Königreich), \gl{nyt}, \gl{smh} und \gl{abc}. 
Die Artikel stammen aus dem Zeitraum vom März bis September 2017 und berichten über Studienergebnisse zur Krebsforschung. 
Dabei wurden nur Originalartikel berücksichtigt, welche die Ergebnisse jeweils einer Studie zusammenfassen. 
Für die Analyse zum Gender Bias haben die Forschenden die Anzahl Autorinnen und Autoren und die Anzahl
männlicher und weiblicher Quellen verglichen.

% Resultate
Die Untersuchung über die 80 Nachrichtenartikel hat ergeben, dass Männer mit 60\% (67/112) bei den \enquote{senior authors}
signifikant übervertreten sind. Dieser Trend ist konstant über alle untersuchten Medienhäuser.
Die Ergebnisse zu den verwendeten Quellen zeigen zudem, dass der Bias bezüglich zitierten Fachpersonen mit 68\% (100/148) Männeranteil
noch stärker ausgeprägt ist. Auch diese Erkenntnis lässt sich über die untersuchten Nachrichtenportale generalisieren, mit der Ausnahme von
\gl{abc}, bei der eine gleiche Anzahl Expertenmeinungen von Männern und Frauen gezählt wurde.

% Diskussion
\citeauthor{cancer-in-the-news} haben einen erheblicher \gl{genderbias} in der Berichterstattung über die Krebsforschung
nachgewiesen. Sowohl bei den Hauptautorinnen und -autoren wie auch bei den Quellen sind Männer deutlich übervertreten. Sie weisen darauf hin, dass der
\gl{mathilda-effect}, der die systematisch kleinere Anerkennung von Wissenschaftlerinnen beschreibt, auch bei der Kommunikation von
wissenschaftlichen Erkenntnissen gilt \cite{rossiter1993matthew,knobloch2013matilda}. Zusätzlich ist die Unterrepräsentation von Frauen in den Medien
im Allgemeinen gut erforscht \cite{10.1371/journal.pone.0148434,ross2011women}, so ihr Hinweis. Der \gl{genderbias} in all seinen Facetten habe
das Potenzial, qualitativ hochstehende Berichterstattung von Forschung durch die Limitierung von Meinungsdiversität zu kompromittieren, argumentieren sie.
Zusätzlich scheint es wahrscheinlich, dass dieser die existierenden Vorurteile verstärkt und die Ungleichheit in der Sichtbarkeit und Anerkennung zwischen
Wissenschaftlerinnen und Wissenschaftlern zementiert, so \citeauthor{cancer-in-the-news}. Dementsprechend betonen sie die Notwendigkeit, den
Effort zur gleichen Repräsentation von Menschen in der Wissenschaft zu verstärken.