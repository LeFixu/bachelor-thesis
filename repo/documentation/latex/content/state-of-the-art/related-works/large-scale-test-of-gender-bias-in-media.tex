\subsection{A Large-Scale Test of Gender Bias in the Media}
Die Autorinnen und Autoren der Arbeit \citetitle{gender_bias_in_media} \cite{gender_bias_in_media} untersuchten den Gender Bias in einem
Datensatz von \enquote{Lydia news analysis system} \cite{lydia}. Dieser enthält zeitgestempelte Nennungen von Personen aus
mehr als 2000 gescannten oder digitalen Zeitungen, Magazinen und online Nachrichtenartikeln aus den Jahren 2004 bis 2009.
Die Quellen stammen vorwiegend aus dem US amerikanischen Raum. 
Sie stellen fest, dass Frauen im Vergleich zu Männern in ähnlichen Positionen und aus denselben Berufsschichten ein grösseres öffentliches 
Interesse geniessen, aber weniger Medienberichterstattung erhalten.

In über 10'000 untersuchten Paaren von Männern und Frauen gleichen Alters, die im selben Sektor tätig sind und gleiches öffentliches
Interesse geniessen, haben die Forschenden einen Unterschied von 46\% in der Anzahl jährlicher Nennungen zugunsten der Männer gemessen.
So wurden die Frauen aus den gebildeten Paaren 300 Mal erwähnt und die Männer 826 Mal.

% Jüngste Studien, in denen grosse Zeitungsdatensätze analysiert wurden, deuten darauf hin, 
%dass der Anstieg des Anteils der Zeitungsberichterstattung über Frauen in letzter Zeit fast zum Stillstand 
%gekommen ist (Gallagher 2010; Macharia 2015). Das Verhältnis von Frauen zu Männern in Zeitungsartikeln hat sich 
%bei etwa 1:4 stabilisiert (Macharia 2015; Shor et al. 2014b, 2015), während das Verhältnis bei Gesichtsbildern 
%zwischen 1:1,5 und 1:4 liegt (Jia et al. 2016). 

Die Forscher:innen argumentieren mit den zwei folgenden Erklärungen wieso die Berichterstattung über Männer unverhältnismässig ist.
Die erste Erklärung könnte die berufliche Ungleichheit zwischen Frauen und Männer sein.
Studien haben gezeigt, dass Männer häufiger vertreten sind in Machtpositionen
\cite{gender_inequality_in_labor_markets,gender_revolution}. 
Besonders in Bereichen, denen die Medien grössere Aufmerksamkeit schenken.
Daraus folgt natürlich, dass die Mainstream-Medien mehr über Männer als über Frauen berichten.

Als zweite Erklärung erkennen die Forscher:innen, dass die Ungleichheit in der Berichterstattung ein unverhältnismässig grosses öffentliches Interesse an Männern widerspiegeln könnte.
Dies kann darauf zurückzuführen sein, dass das Publikum Nachrichten über Männer bevorzugt oder dass Männer ein grösseres Interesse wecken, 
weil sie Verhaltensweisen an den Tag legen, die die Öffentlichkeit für berichtenswert hält.

% Die Forscher:innen versuchen, die folgenden drei Forschungsfragen zu beantworten:
% \begin{enumerate}
%     \item Erhalten Frauen weniger Berichterstattung als Männer, selbst wenn sie gleichwertige Positionen oder Leistungen in Politik, Wirtschaft, Unterhaltung, Sport und anderen sozialen und beruflichen Bereichen erreicht haben? 
%     \item Unterscheidet sich das öffentliche Interesse an Männern und Frauen, die eine ähnliche Position erreicht haben? 
%     \item Erhalten Frauen selbst dann weniger Medienaufmerksamkeit als vergleichbare Männer, wenn man sowohl ihre strukturelle Positionierung als auch das öffentliche Interesse berücksichtigt, das sie auf sich ziehen?
% \end{enumerate}

% Why Might Successful Women Receive Less Coverage Compared With Successful Men? Causal Mechanisms and Research Hypotheses

Die Forscher:innen gehen davon aus, dass Ungleichheiten in der Medienberichterstattung über Männer und Frauen in erster Linie 
auf eine der folgenden drei Hauptkategorien von Erklärungen zurückzuführen sind:
\begin{itemize}
    \item strukturelle Ungleichheiten in der Arbeitswelt und in hochrangigen Positionen 
    \item Ungleichheiten in Bezug auf öffentliches Interesse und Nachrichtenwert bei gleichwertigen Positionen
    \item Medienspezifische Faktoren: Voreingenommenheit und Diskriminierung in der Medienberichterstattung
\end{itemize}

\subsubsection{Strukturelle Ungleichheiten in der Arbeitswelt und in hochrangigen Positionen}
Frauen haben oft einen eingeschränkten Zugang zu hochrangigen Berufen und Positionen und werden durch Stereotypen und Diskriminierung benachteiligt \cite{glass_ceiling_politics}.
Frauen stossen häufig auf eine "gläserne Decke", die ihr Fortkommen in der Arbeitswelt behindert
\cite{glass_ceiling_effect,glass_ceiling_politics,president_glass_ceiling}.
Obwohl diese Ungleichheiten die Berichterstattung von Männern begünstigen, bleibt unklar, ob sie allein für 
den Unterschied in der Medienpräsenz von Männern und Frauen verantwortlich sind. 
Die Forscher:innen denken an, Frauen und Männer in ähnlichen Positionen oder mit ähnlichen Errungenschaften zu vergleichen, 
um zu bestimmen, ob strukturelle Ungleichheiten allein für die Unterschiede in der Berichterstattung verantwortlich sind.

\subsubsection{Ungleichheiten in Bezug auf öffentliches Interesse und Nachrichtenwert bei gleichwertigen Positionen}
Die Forscher:innen weisen darauf hin, dass selbst ein systematischer Vergleich von Männern und Frauen, die gleichwertige Positionen 
in verschiedenen sozialen und beruflichen Teilbereichen erreicht haben, nicht ausreicht. 
Denn auch das öffentliche Interesse muss berücksichtigt werden. 
Die Medien sind mit kommerziellen Interessen und Zwängen konfrontiert und müssen Leser:innen und Zuschauer:innen Quoten berücksichtigen.
Es stellt sich die Frage, ob es systematische Unterschiede im öffentlichen Interesse zwischen Männern und Frauen in 
vergleichbaren Positionen gibt.
%Die Literatur über geschlechtsspezifische Ungleichheiten, insbesondere am Arbeitsplatz, bietet eine Reihe von Mechanismen, 
%durch die das Geschlecht mit dem öffentlichen Interesse interagieren und es beeinflussen kann. 
Einerseits könnten Frauen, die hohe soziale und berufliche Positionen erreicht haben, im Vergleich zu ihren männlichen Kollegen 
mehr Interesse auf sich ziehen, da sie aufgrund von Barrieren und Diskriminierung besonders talentiert und 
erfolgreich sein müssen \cite{glass_ceiling_politics}. %(Folke und Rickne (2016))
Andererseits deuten feministische und geschlechtsspezifische Theorien darauf hin, dass Frauen tatsächlich 
weniger öffentliches Interesse auf sich ziehen könnten, selbst wenn sie führende soziale und 
berufliche Positionen erreichen \cite{comparative_sociology,status_matters_for_inequality,social_difference_status_distinction}.  %(Lamont 2012; Ridgeway 2013; Ridgeway et al. 2009)
Die Gründe dafür liegen in fest verwurzelten kulturellen Überzeugungen über Geschlechterstatus und die Arten von Personen, 
die besonderen Respekt verdienen \cite{status_matters_for_inequality}. %(Ridgeway 2013)

\subsubsection{Medienspezifische Faktoren: Voreingenommenheit und Diskriminierung in der Medienberichterstattung}
Ähnlich wie bei geschlechtsspezifischen Lohnunterschieden können Unterschiede, die nicht auf strukturelle 
Ungleichheiten zurückzuführen sind, als geschlechtsspezifische Verzerrungen betrachtet werden 
\cite{gender_stereotyping,gender_gap_lawyers}. % (Castagnetti und Rosti 2013; Dinovitzer, Reichman und Sterling 2009)
Die Verzerrung kann Frauen sowohl begünstigen als auch benachteiligen. 
Sie kann durch Statusvorurteile entstehen, die die Bereitschaft beeinflussen, Frauen Aufmerksamkeit 
zu schenken und ihre Leistungen positiv zu bewerten \cite{status_matters_for_inequality,unpacking_gender_system,social_difference_status_distinction}. % (Ridgeway (2013), Ridgeway und Correll (2004) sowie Ridgeway et al. (2009))
Die täglichen Entscheidungen von Journalist:innen, Redakteur:innen und Verleger:innen, die in einem männlich geprägten 
kulturellen Umfeld arbeiten, tragen zur Verschärfung der Ungleichgewichte in der Berichterstattung 
bei und verzerren sie zugunsten von Männern. 
%Frauen erhalten oft nicht das gleiche Mass an Berichterstattung wie ihre männlichen Kollegen. 
Die historische Dominanz von Männern in redaktionellen Positionen hat eine Machtstruktur geschaffen, 
die eine männlich geprägte Berichterstattung und redaktionelle Normen diktiert \cite{female_reporting,women_and_news}. % (Rodgers und Thorson 2003; Ross und Carter 2011)


% Conclusion and Discussion
\subsubsection{Schlussfolgerung und Diskussion}
Die Forscher:innen untersuchten mehr als 20'000 namhafte Männer und Frauen aus verschiedenen sozialen und 
beruflichen Bereichen und verglichen die Medienberichterstattung und das öffentliche Interesse. 
Die Studie ergab, dass Männer trotz des grösseren öffentlichen Interesses an Frauen mehr als doppelt so oft erwähnt wurde. 
Nach einem Abgleich von Männern und Frauen nach Erfahrung, struktureller Position und öffentlichem Interesse, 
haben die Zeitungen den Männern in fast allen Bereichen einen deutlich grösseren Umfang an Berichterstattung zugestanden. 
Auch wenn Frauen bedeutende Leistungen vollbringen und zweifellos im öffentlichen Fokus stehen, erhalten sie normalerweise 
weniger Medienaufmerksamkeit als Männer mit ähnlichem öffentlichem Interesse.
Die Studie zeigt, dass theoretische Vorschläge, die entweder strukturelle Ungleichheiten oder individuelle 
Unterschiede in Motivation, Anstrengung, Talent oder Charisma hervorheben, unzureichend sind, wenn es darum 
geht, die geschlechtsspezifischen Unterschiede in der Berichterstattung zu erklären. 
Die Forschung zu geschlechtsspezifischen Ungleichheiten auf dem Arbeitsmarkt hat gezeigt, dass Frauen beim 
Berufseinstieg und beim Aufstieg innerhalb ihrer Berufe aufgrund von strukturellen Hindernissen benachteiligt 
sind und somit in der Öffentlichkeit weniger präsent sind \cite{glass_ceiling_effect,glass_ceiling_politics,president_glass_ceiling}. 
Die vorliegende Studie signalisiert jedoch, dass diese strukturellen Ungleichheiten die Lücke nicht vollständig 
erklären können, die selbst beim Vergleich von Männern und Frauen, die ähnliche Positionen und Leistungen 
erreicht haben, erheblich bleibt. 
Es gibt Hinweise darauf, dass Frauen als weniger wertvoll für die öffentliche Aufmerksamkeit angesehen werden. 
Dies ist sowohl auf kulturelle Überzeugungen zurückzuführen, die Frauen und ihre Handlungen 
abwerten, als auch darauf, dass Frauen eine kürzere Karrieredauer in prominenten Ämtern haben, ihnen 
weniger wichtige Positionen zugewiesen werden und sie sich seltener selbst vermarkten \cite{status_matters_for_inequality,social_difference_status_distinction}. 
Die Studie leistet somit einen wichtigen Beitrag zur Massenkommunikationsliteratur über geschlechtsspezifische 
Unterschiede in der Medienberichterstattung und zur umfassenderen Literatur über kulturelle und berufliche 
Ungleichheiten zwischen den Geschlechtern.
