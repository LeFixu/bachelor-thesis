\subsection{Using citation network analysis to enhance scholarship in psychological science: A case study of the human aggression literature}

In dem Forschungsartikel \citetitle{citation-network} untersuchten \citeauthor{citation-network} mithilfe einer graphbasierten Analyse
die Zusammensetzung der Forschungsgemeinschaft zum Thema \enquote{Aggression} \cite{citation-network}. Neben der Untersuchung zu den
einflussreichsten Gemeinschaften auf diesem Feld wollten die Forscher:innen ausserdem herausfinden, ob ihre Netzwerkanalyse dazu beitragen kann,
den \gl{genderbias} beim Zitieren von wissenschaftlichen Arbeiten zu reduzieren.

Der Grund für diesen Bias, so mutmassen die Autorinnen und Autoren des Artikels, liegt in der Art wie die Relevanz von wissenschaftlichen Arbeiten beurteilt wird.
So werden aktuell diejenigen Papers als am relevantesten eingestuft, die am meisten zitiert wurden. Dieser Fokus auf der Anzahl Zitierungen begünstigt
vermutlich einen Bias zum Vorteil von etablierten Forschungsgemeinschaften und führt zu einer Unterrepräsentation von anderen wichtigen Forschungsgruppen,
so \citeauthor{citation-network}. Dass dieser Ansatz im Falle von Publikationen im Bereich der Astronomie \gl{genderbias} begünstigt, haben
\citeauthor{citation-astronomy} bewiesen \cite{citation-astronomy}.

In ihrem Experiment vergleichen \citeauthor{citation-network} den Frauenanteil unter den Autorinnen und Autoren aus den Top 75 Artikeln, ausgewählt nach drei Methodiken.
\begin{enumerate}
    \item Anzahl Zitierungen
    \item Kompositrangierung
    \item Kompositrangierung mit Einbezug der Gemeinschaften
\end{enumerate}

Die erste Methodik, \enquote{Anzahl Zitierungen}, entspricht dem Status Quo. Also, dass die wichtigsten Arbeiten anhand der absoluten Anzahl der Zitierungen
identifiziert werden. Die zweite Methodik nennen die Forschenden \enquote{composite score}. Auf Deutsch wäre das eine \enquote{Kompositrangierung}.
Diese besteht aus sechs Teilbewertungen (Komposite) zur Messung der Zentralität einer Arbeit im gesamten Zitatnetzwerk.
Der letzte Ansatz ist eine Erweiterung der Kompositrangierung und teilt allen Gemeinschaften im Graphen gleich viele Ranglistenplätze zu. Diese werden dann anhand der
\enquote{Kompositrangierung} sortiert.

Die Auswertung betrachtet die 75 bestplatzierten Arbeiten und vergleicht den Anteil der Autorinnen untereinander. Während dieser gemessen an der traditionellen Methode
gerade mal 16\% (12) beträgt, sind es bei der \enquote{Kompositrangierung} bereits 20\% (15) und unter Berücksichtigung der Gemeinschaften gleich 32\% (24).
Die \enquote{Kompositrangierung mit Einbezug der Gemeinschaften} hat im verwendeten Datensatz also zu einer Verdoppelung der Quote von Autorinnen geführt. 
\citeauthor{citation-network} bemerken, dass diese Art der Relevanzbewertung nicht nur dabei helfen kann, unterrepräsentierte Gemeinschaften in der Forschung besser 
sichtbar zu machen, sondern auch den \gl{gendergap} zu reduzieren.

Leider bietet die Arbeit keine Auskunft zum Anteil der Autorinnen über das gesamte Datenset, was das Einordnen ihrer Zahlen erschwert. So ist es nicht möglich zu beurteilen,
ob der Anteil der Arbeiten, die von Frauen verfasst wurden, insgesamt mehr oder weniger als 32\% ausmacht und wie gross oder klein der \gl{gendergap} damit noch immer ist.