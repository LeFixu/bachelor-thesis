\subsection{DIANES: A DEI Audit Toolkit for News Sources}
Die Autoren und Autorinnen der Arbeit \citetitle{10.1145/3477495.3531660} \cite{10.1145/3477495.3531660} haben ein Audit Toolkit entwickelt,
um Journalist:innen und Reporter:innen von Nachrichtenportalen dabei zu helfen, 
den Anteil der Quellenvielfalt (z.B. Geschlecht und Rasse) für Zitate in ihren Artikeln zu verbessern.
Laut den Verfasser:innen der Arbeit wird in professionellen Nachrichtenmedien immer betont, wie wichtig die Vielfalt der Perspektiven ist, 
aber in der Praxis wird dem nur wenig Beachtung geschenkt.
Daher sind die Nachrichtenmedien unter ethische Kritik geraten im Rahmen der neuen Normen für Vielfalt, Gleichberechtigung und Integration, 
im englischen \gl{dei}.
Das Toolkit soll die Medienschaffenden dabei unterstützen, diesen Normen mehr Beachtung zu schenken.
\gl{dianes} ist ein \gl{dei} Toolkit, das eine NLP-Pipeline im Backend zur Extraktion von Zitaten, Sprechern, Titeln und Organisationen
aus Nachrichtenartikeln verwendet und diese in Echtzeit verarbeitet. 

Im Frontend unterstützt \gl{dianes} eine Vielzahl von Benutzeranfragen zum \gl{dei} Audit auf der Grundlage der im Backend extrahierten Informationen. 
Auf dem Artikel Dashboard können Medienschaffende ihren Artikeltext als Entwurf eingeben und 
dann die entsprechenden Informationen zur Quellenvielfalt in diesem Artikel abrufen.
Das Dashboard der Quellenvielfalt ist dazu da, die \gl{dei} Daten für einen ausgewählten Zeitraum von allen Artikeln zu visualisieren.

Das \gls{api} implementiert die Möglichkeit, alle Zitate, Namen, Geschlechter sowie Titel der Sprecher:innen zu extrahieren, 
die in Artikeltexten vorkommen, welche der \gls{api} gesendet werden.

% Backend Text
Das Backend Modul nimmt den Nachrichtenartikel als Input und extrahiert die Zitate und Sprecher:innen mit ihren Titeln und
prognostiziert dann Geschlecht und Rasse/Ethnie dieser zitierten Person.
Das Modul ist in folgende vier Stages unterteilt, von welchen die ersten zwei Stages  
die Standford CoreNLP Bibliothek \footnote{https://stanfordnlp.github.io/CoreNLP/} nutzen.

\subsubsection{Text Processing}
In der \enquote{Text Processing} Phase werden zunächst unnötige Informationen aus dem Text gefiltert wie \ashort{xml} Tags.
Nach dem Filtern wird der Text tokenisiert und im nächsten Schritt wird \gl{pos} Tagging angewendet um z.B. zu bestimmen,
ob die einzelnen Wörter (Tokens) Verben oder Substantive sind. 
Danach wird eine \gl{lemming} durchgeführt, um jedes Wort auf seine Grundform zu reduzieren.

\subsubsection{NLP Analysis}
Der \enquote{NLP Analysis} Schritt enthält die folgenden spezifischen Komponenten.
Die Satzspaltung teilt den Text in Sätze auf. Dies wird anhand des CoreNLP Regelwerks gemacht.
Mit \gl{ner} werden Personen, Titel und Organisationen im Text bestimmt.
Dependency Parsing analysiert die grammatikalischen Beziehungen zwischen Wörtern in einem Satz 
und extrahiert textuelle Beziehungen auf Basis der Abhängigkeiten.
Die \gl{cr} findet Erwähnungen der gleichen Entität in einem Text, z.B. wenn 
sich \enquote{Anne} und \enquote{sie} auf dieselbe Person beziehen.
\gl{kbp} wird verwendet um Titel und Organisationen einer Person zu finden, wenn diese
im Text vorkommen. 

\subsubsection{Annotations}
In diesem Teil werden Annotationen erstellt für Zitate, Sprecher:innen und ihre Titel und 
Organisationen auf der Grundlage der vorherigen \gl{nlp} Analyse. 
Wenn bei einem Zitat ein Anführungszeichen oder Schlusszeichen fehlt, hat CoreNLP den Text bis 
zum nächsten Anführungszeichen/Schlusszeichen des nächsten Zitates als Zitat Inhalt deklariert, was die Genauigkeit der Zitat-Auflösung verringern kann.
Um dieses Problem zu umgehen, arbeiteten die Entwickler:innen mit der Anzahl Wörter in vorgängigen Zitaten.
Zitate mit über 100 Wörter oder unter fünf Wörter wurden für die Auswertung ausgeschlossen.
Die Entwickler:innen machten die Erfahrung, dass CoreNLP Probleme hatte und oft die falsche Person zum Zitat zuwies.

\subsubsection{Representation Recognition:}
Diese Phase dient dazu Geschlecht und Rasse/Ethnie der zitierten Person zu definieren.
Diese zwei Repräsentationsattribute von \gl{dei} sind die entscheidenden Informationen zum Verständnis der Repräsentativität der Nachrichtenbeschaffung von \gl{dianes}.
Das Geschlecht wurde mit der Gender API \footnote{https://gender-api.com/} bestimmt.
Es gibt viel weniger existierende Arbeiten zur Vorhersage der Rasse/Ethnizität
einer Person mit einem Namen, und die Aufgabe ist anspruchsvoller als Geschlechtserkennung.
Darum wurde zur Vorhersage der Rasse/Ethnizität einer Person ein eigenes \gl{ml} Model implementiert und trainiert, 
welches als Input den Namen der Person braucht.
Der Datensatz, der zum Trainieren des Rassendetektors verwendet wurde, stammt von dem United States Census Bureau aus den Jahren 2000 und 2010. 
Er enthält insgesamt 151'670 eindeutige Namen in sechs Kategorien (White, African, American, American Indian, Alaska Native, Asian, Native
Hawaiian und Other Pacific Islander).

\gl{dianes} wurde zum Zeitpunkt der Veröffentlichung der Arbeit von mehreren Redaktionen in der Praxis getestet.