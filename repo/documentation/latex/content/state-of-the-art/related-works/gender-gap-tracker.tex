\subsection{The Gender Gap Tracker: Using Natural Language Processing to measure gender bias in media}

Die wissenschaftliche Arbeit \citetitle{gender_gap_tracker} \cite{gender_gap_tracker} von \citeauthor{gender_gap_tracker}
aus dem Jahr \citeyear{gender_gap_tracker} ist das Vorbild für die vorliegende Thesis. Im Rahmen
ihrer umfassenden Untersuchung der kanadischen, englischsprachigen Medien haben die Verfasser:innen nicht nur
den Gender Gap anhand der Anzahl Zitate pro Geschlecht und Nachrichtenportal untersucht, sondern 
auch neue Standards zur Messung und kontinuierlichen Überwachung der Repräsentation von Männern und Frauen in den
Medien geschaffen.

Aufgrund der ausführlichen Beschreibung ihrer Vorgehensweise und der geringen Eintrittshürde in
Bezug auf notwendiges vorprozessiertes Datenmaterial, kann diese Methodik auch gut in anderen Kultur-
und Sprachräumen nachgeahmt werden. Da dieses Projekt genau das vorhat, ist diese Arbeit in einem
grossen Detailgrad beschrieben.

\subsubsection{Ziele der Arbeit}

Das Ziel der Arbeit war es, den Unterschied in der Repräsentation der Geschlechter in den kanadischen
Medien sichtbar zu machen. Als Mass für die Repräsentation verwenden \citeauthor{gender_gap_tracker}
die Anzahl der Zitate in den Online-Nachrichtenportalen, die von Männern und Frauen publiziert werden.

Dazu haben sie eine Applikation entwickelt, die auf regelmässiger Basis die neusten Artikel
der Portale mittels Webcrawling herunterlädt und analysiert. Für die Untersuchung der Texte verwenden sie
verschiedene Arten von \gl{nlp}, über welche die folgenden Unterkapitel mehr Aufschluss geben sollen.

Schlussendlich visualisieren sie die Daten auf einer öffentlich zugänglichen Plattform in der Form einer
interaktiven Webapplikation
\footnote{https://gendergaptracker.informedopinions.org/}.
Die Resultate aus der Arbeit stammen direkt von dieser Software.

Die stets aktuellen Daten des \gl{ggt} werden bewusst öffentlich zur Verfügung gestellt, um den Medienschaffenden
Anreize zu verschaffen, für mehr Diversität in ihrer Berichterstattung zu sorgen. In der Arbeit
vergleichen sie den \gl{ggt} mit einem Fitness Tracker, der den Menschen helfen soll für ihrer
Fitness zu sorgen. Analog zu diesem Prinzip soll dieses Tool den Medienkonzernen mithilfe von Transparenz
und Rückverfolgbarkeit helfen, ihren \gl{genderbias} zu verringern.

\subsubsection{Zitate extrahieren}\label{ggt-method-citation}

Der Kern der Aufgabe des \gl{ggt} ist das Extrahieren von Zitaten. Die Entwickler:innen haben sich dazu eines vortrainierten, syntaktischen Parsers von Spacy bedient,
der Parsetrees aus Sätzen generieren kann. Die Software teilt die zu analysierenden Artikel also in die einzelnen Sätze auf und analysiert mit dem genannten Parser deren
einzelne Wörter. Mithilfe dieser Wörter und der dazugehörigen Bedeutung, die das Modell ihnen zuschreibt, kann die Applikation im Anschluss
zwei unterschiedliche Arten von Zitaten erkennen. Diese sind auch für die vorliegende Arbeit
von fundamentaler Bedeutung und wurden deshalb im Kapitel \ref{types-of-citations} im
Detail beschrieben.
\begin{enumerate}
    \item \gl{quote-syntactic}
    \item \gl{quote-floating}
\end{enumerate}

Diese Art Zitate zu unterscheiden erscheint auf den ersten Blick nicht intuitiv. So werden Zitate ja entweder in direkter, also mit Anführungs-
und Schlusszeichen, oder in indirekter Rede verfasst. Der naheliegende Ansatz wäre also die Unterscheidung nach diesen zwei Mustern. \citeauthor{SpronckNikitina+2019+119+159}
haben aber herausgefunden, dass ein syntaktischer Ansatz zum Parsen (mithilfe der Satzstruktur) bessere Resultate liefert, als semantisches Parsing (anhand von semantischen Merkmalen
wie Satzzeichen oder bestimmten Wörtern) \cite{SpronckNikitina+2019+119+159}. Aufgrund dieser Erkenntnis orientieren sie sich am syntaktischen Ansatz.

\subsubsection{Personen identifizieren und deren Geschlecht bestimmen}

Für ihre Auswertungen extrahieren die Forscher:innen alle Personen aus den Artikeln und bestimmen ihr Geschlecht.
Insbesondere interessieren sie sich für
\begin{enumerate}
    \item die Autorinnen und Autoren der Texte,
    \item die zitierten Personen (Quellen)
\end{enumerate}
aber auch generell für alle anderen Menschen, die im Text erwähnt werden.

Um Personen in den Artikeln zu finden, bedienen sie sich eines \textsl{\acrshort{ner}s (\textsl{Named Entity Recognizers})}. Dieser spezialisierte Parser kann aus Texten
Entitäten wie Personen, Orte und Firmen extrahieren. In diesem Fall werden nur die Personen weiterverarbeitet.

Das Resultat dieses Verfahrens ist eine Liste mit allen Wörtern und Satzteilen, die Personen repräsentieren. Diese Liste ist unbereinigt und
wird mithilfe eines \gl{cr} Algorithmus in zusammengehörige Gruppen aufgeteilt. So wird der \gl{ner} in einem Text über den
Bundesrat Berset viele unterschiedliche Repräsentationen dieses Mannes finden, wie zum Beispiel \enquote{Bundesrat Berset}, \enquote{er}, \enquote{Alain Berset}
etc. Da diese Nennungen alle der gleichen Person gelten, ist es für die Auswertung relevant, diese zu clustern.

Zur Identifikation des Geschlechts verwendet die Applikation eine Web API, die entweder nur mithilfe des Vornamens, oder mit Vornamen und Nachnamen, das Geschlecht
bestimmen kann. Einzelne Fälle, welche die API falsch zuordnet, führt die Applikation in einer dedizierten Liste.

\subsubsection{Evaluation}

Zur Qualitätssicherung haben die Forschenden kontinuierlich Messungen zur Performance ihrer Auswertungen durchgeführt. Das ermöglichte es ihnen, den Effekt von implementierten Verbesserungen
direkt messen zu können. Als Grundlage dazu dienten 14 repräsentative, manuell ausgewertete Artikel aus allen sieben verwendeten Nachrichtenportalen mit einer Mindestlänge von 3000 Zeichen.
Mithilfe dieses Datensatzes konnten sie die \gl{precision}, den \gl{recall} und den \gl{f1-score} für die Extraktion von Zitaten, Personen und Verben bestimmen.

\subsubsection{Resultate}
% Absoluter Gender Gap
Die Resultate der Arbeit können einen klaren \gl{gender-bias-distribution} in den untersuchten
Onlineartikel aus sieben kanadischen Medienhäusern feststellen. So sind die Zitate in ihrem Datensatz
in einem Durchschnitt über zwei Jahre (01.10.2018 - 30.09.2020) in nur 29\% der Fälle Frauen zuzuordnen.
Die verbleibenden 71\% sind Zitate von Männern.

Der Unterschied in der Verteilung der Zitate zwischen den Geschlechtern variiert zwischen den Medienhäusern,
wobei CBC News mit 33\% der Zitate am besten und The Globe and Mail mit 23\% am schlechtesten abschneiden.

Aus ihren Daten ist ersichtlich, dass Frauen während der Coronapandemie häufiger zitiert wurden als zuvor.
Die Autorinnen und Autoren weisen auf die Tatsache hin, dass Frauen in den Gesundheitsämtern häufiger vertreten sind als in
anderen Bereichen der Arbeitswelt und dass eine ausgeglichene Repräsentation der Geschlechter in wichtigen
Ämtern den \gl{gendergap} zu verringern scheint.

% Gender Gap unter wichtigen Personen
Die für diese Auswertung gesammelten Daten boten auch die Möglichkeit, Untersuchungen zu den am häufigsten zitierten
Menschen zu machen. So haben die Forscher:innen einen Ausschnitt der Top 15 zitierten Personen aus dem Datenset genauer
untersucht. Diese haben sie im Anschluss Kategorien wie Politik, öffentlicher Gesundheit und Privatwirtschaft zugewiesen.
Die meistzitierte Frau in dieser Liste würde mit 2'239 Zitierungen gerade mal den 8 Platz in der Rangliste der Männer
einnehmen. Interessanterweise sind alle 15 Vertreter der meistzitierten Männer im Sektor der Politik tätig. Auch bei den
Frauen sind die meisten Politikerinnen. Drei sind aus dem Bereich der öffentlichen Gesundheit und eine aus der Privatwirtschaft.
Dass es drei Vertreterinnen aus dem Gesundheitssektor in die Top 15 meist zitierten Frauen im Datensatz schaffen, weist ebenfalls
darauf hin, dass eine ausgeglichenere Frauenquote in für die Öffentlichkeit interessanten Bereichen den Geschlechtergraben verkleinern
könnte.


% Gender Gap nach Bereich der Gesellschaft
Eine weitere Analyse der Wissenschaftler:innen zeigt die Aufteilung der Zitate nach Geschlecht und Tätigkeitsgebiet der Quelle.
Die tabellarische Auswertung zeigt, dass die meistzitierten Männer Politiker (73.8\%), Sportler (7.7\%) und Regierungsangestellte sind.
Bei den Frauen finden sich die meisten Zitate in den Tätigkeitsgebieten Politik (52.6\%), Gesundheit (16.7\%) und Exekutive (8.3\%) (Regierungsangestellte).
Auch hier fällt wieder auf, wie massiv weniger Frauen zitiert werden. So werden Politiker 103'378 und Politikerinnen 29'007 Mal zitiert,
Sportler 10'723 und Sportlerinnen 1'415 Mal und Regierungsangestellte Männer 9'175 und Frauen 4'583 Mal. Nur in den Bereichen öffentliche Gesundheit,
NGO's und Wissenschaft werden Frauen leicht häufiger zitiert.

Eine interessante Nebenerkenntnis dieser Auswertung ist die Tatsache, dass insgesamt weniger Zitate von Frauen in Online-Nachrichtenportalen
erscheinen als von Männern, jedoch absolut gesehen mehr individuelle Frauen zitiert werden. Das bedeutet, dass die Anzahl Zitate pro Mann
deutlich höher ist als die Anzahl Zitate pro Frau.

Sie erkennen das Muster, dass Männern mehr Raum gegeben wird als Frauen und einige wenige Kategorien den grössten Teil der Berichterstattung
dominieren. Die Forscher:innen spekulieren aufgrund dieser Erkenntnis, dass die Daten einer Pareto Verteilung unterliegen. Also dass das
grösste Hindernis mehr von Frauen zu hören die Bevorzugung derer ist, die bereits sehr viel Raum einnehmen.

% Gender Gap nach Geschlecht des Authors
In einer weiteren Auswertung haben \citeauthor{gender_gap_tracker} den Zusammenhang zwischen dem Geschlecht des
Autors oder der Autorin und der Anzahl und Verteilung der Zitate von Männern und Frauen untersucht. Sie wollten
damit herausfinden, ob Frauen häufiger weibliche Quellen zitieren als Männer. In der Tat scheinen Journalistinnen
häufiger Frauen zu zitieren als Journalisten. Während Männer im Schnitt nur in einem Viertel (25\%) ihrer Zitate auf Frauen
zurückgreifen, sind es bei den Frauen immerhin 34\%. Daraus schliessen die Autorinnen und Autoren, dass ein Teil der Lösung zum
Gender Gap in der Erhöhung der Quote von Reporterinnen besteht. Nicht nur, weil sie in selbst geschriebenen Artikeln
häufiger Frauen zitieren, sondern weil sie auch in gemischten Teams einen positiven Einfluss auf die Wahl der Quellen
ausüben, wie die Auswertung zeigt.

% Applikation
Die Applikation, die auf täglicher Basis die für diese Auswertungen notwendigen Daten aggregiert, ist nach wie vor im Einsatz und wird - wenn auch sporadisch, 
doch immer wieder weiterentwickelt. Die aktuellsten Auswertungen sind in interaktiver Form auf \href{https://gendergaptracker.informedopinions.org/
}{GenderGapTracker.InformedOpinions.org} zu finden. Der Code dazu ist öffentlich auf \href{https://github.com/sfu-discourse-lab/GenderGapTracker}{GitHub} einsehbar.


\subsubsection{Weiterentwicklung des Gender Gap Trackers}
Nach der Fertigstellung des Papers im Jahr \citeyear{gender_gap_tracker}, wurde die Applikation um die Möglichkeit, auch französische Artikel
zu analysieren, erweitert. Damit analysiert der \gl{ggt} zum aktuellen Stand sieben englisch- und sechs französischsprachige Nachrichtenportale.

Da die Release Notes auf GitHub erst seit September 2022 geführt werden, bleiben wohl viele neue Features, die in der Arbeit nicht beschrieben sind,
verborgen. Die Entwickler:innen sind jedoch fleissig am \gl{refactorn} und scheinen den \gl{ggt} regelmässig zu warten und weiterzuentwickeln.