\subsection{The gender gap in commenting: Women are less likely than men to comment on (men's) published research}
Die Studie \citetitle{gender_gap_in_commenting} \cite{gender_gap_in_commenting} behandelt den \gls{gendergap}
im Kommentieren von veröffentlichten Forschungsartikeln. 
Die Autorinnen und Autoren gehen der Frage nach, ob auch das Kommentieren von veröffentlichten Forschungsartikeln geschlechtsspezifisch ist.
In der Studie haben die Verfasser:innen anhand von 1'350 Kommentaren zu 1'236 Forschungsartikeln, 
welche in den Fachzeitschriften - \gl{pnas} und Science - veröffentlicht wurden festgestellt,
dass es eine geschlechtsspezifische Diskrepanz bei den Verfasser:innen von Kommentaren gibt.

Die Autorinnen und Autoren haben sich gefragt, wieso Frauen seltener Kommentieren als Männer.
Sie haben die Theorie aufgestellt, dass es beim Kommentieren um Herausforderung, Risiko und Belohnung geht.

\subsubsection{Allgemeine Risikoaversion}
Wenn eine Person sich entscheidet, einen für alle sichtbaren Kommentar unter einen veröffentlichten Forschungsartikel zu schreiben,
geht sie auch das Risiko ein, den eigenen Ruf zu schädigen.
Das Kommentieren setzt die Bereitschaft voraus, die Überlegenheit der eigenen wissenschaftlichen Erkenntnisse über die eines anderen zu behaupten.
Die Forscher:innen treffen die Annahme, dass Wissenschaftler:innen mit der Abgabe von Kommentaren warten, 
bis sich ihre Karriere besser etabliert hat.
Da Frauen meist auf tieferen Karriere Stufen sind als Männer, verfassen sie deshalb weniger Kommentare, so ihre Theorie.
In diesem Fall wäre der Unterschied bei der Verteilung der Geschlechter im Kommentieren von Forschungsartikeln ein Pipeline-Effekt.

\subsubsection{Geschlechtsspezifische Risikoaversion}
Möglicherweise könnte das Geschlecht die relative Abwägung der Risiken und der Chancen durch einen Kommentar beeinflussen.
Aus jahrzehntelanger Forschung ist bekannt, dass Männer risikofreudiger sind als Frauen 
\cite{gender_differences_in_risk_taking,gender_and_culture,self_promotion_risk_factor_for_women,gender_differences_automatic_in-group_bias,gender_differences_in_risk_assessment}. % 38
Daraus ist zu schliessen, dass sich Frauen mehr Sorgen über die negativen Folgen machen, wenn sie einen Forschungsartikel kommentieren wollen
und dies darum seltener tun.
Die Autorinnen und Autoren erwarten auch, dass die geschlechtsspezifischen Unterschiede bei der Kommentierung schwächer ausfallen, 
wenn der Zielartikel von einer jüngeren Person verfasst wurde und somit das Risiko, sich gegen sie zu stellen, geringer ist.

\subsubsection{Geschlechtsspezifisches Risiko}
Denkbar ist auch, dass die Folgen von Kommentaren geschlechtsspezifisch sind.
Frauen und Männer werden für dasselbe Verhalten unterschiedlich bestraft und belohnt.
Von Frauen wird erwartet, dass sie gemeinschaftlich und nicht durchsetzungsfähig sind
\cite{burgess1999women,eagly2002role}. % 40
Bei Männer hingegen hat die Gesellschaft die Erwartung, dass sie durchsetzungsfähig und wettbewerbsorientiert sind.
Ein Verstoss gegen diese geschlechtsspezifischen Stereotypen führt in der Regel zu Missbilligung
\cite{self_promotion_risk_factor_for_women,mclaughlin2012sexual,mark_of_a_womans_record,rudman2008backlash}. % 43
Wenn Frauen als durchsetzungsfähig wahrgenommen werden und in männerdominierten Bereichen Erfolg haben, 
sinkt ihre \enquote{Beliebtheit} in der Regel mit negativen sozialen und wirtschaftlichen Folgen
\cite{self_promotion_risk_factor_for_women,mclaughlin2012sexual,mark_of_a_womans_record,rudman2008backlash}. % 43
Während Kommentare im Allgemeinen für Frauen unverhältnismässig riskant sein könnten, 
dürfte dies noch ausgeprägter sein, wenn es sich bei der Zielperson um einen männlichen Wissenschaftler handelt, 
da dies die vermeintliche Überlegenheit von Männern infrage stellt. 
Dies bedeutet, dass Frauen im Vergleich zu männlichen Kommentatoren wahrscheinlich weniger Artikel angreifen, 
die von Männern verfasst wurden.

\subsubsection{Geschlechtsspezifische Fürsorge unterdrückt Herausforderungen}
Frauen gelten oft als das fürsorglichere Geschlecht \cite{marsh2019caring}. % 46
Das Kommentieren ist nicht unbedingt eine angenehme Praxis, da es die Arbeit eines anderen infrage stellt. 
Es ist durchaus denkbar, dass potenzielle Verfasser:innen von Kommentaren befürchten, 
dass ihr Kommentar negative Folgen für die Betroffenen haben könnte.
Dies könnte die Quote der Kommentare von Frauen senken.

%Die Resultate der Untersuchung der Kommentare zeigt, dass nur 15 \% der Kommentare eine weibliche Erstautorin haben 
%gegenüber 26 \% der Artikel welcher von Frauen verfasst wurden, ein signifikanter Unterschied (p < .001).
Die Resultate der Untersuchung zeigen, dass unter den jüngeren, gefährdeteren Wissenschaftler:innen der Anteil der Autorinnen 
sowohl bei Kommentaren als auch bei Artikeln höher ist, wahrscheinlich weil Frauen in den unteren akademischen Rängen stärker vertreten sind.
%Zwar ist der geschlechtsspezifische Unterschied bei der Verfasserschaft von Kommentaren und Artikeln bei denjenigen, 
%die keine korrespondierenden Autoren sind, tatsächlich größer, wie es das Argument der geschlechtsspezifischen Risikoaversion vorhersagt, 
%doch ist dieser Unterschied auf konventionellem Niveau nicht signifikant (p = .09).

% noch mal schauen ob das nicht zu viel infos sind
Wenn eine grössere Risikoaversion der Grund für die geringere Rate an Kommentaren von Frauen ist, 
erwarten die Forscher:innen auch eine geringere geschlechtsspezifische Diskrepanz bei der Verfasser:innenschaft von Kommentaren, 
wenn der Erstautor relativ machtlos ist, da dies die Risiken beim Kommentieren im Allgemeinen weniger auffällig machen sollte. 
Umgekehrt, wenn Frauen insgesamt weniger kommentieren, weil sie sich mehr Sorgen machen, 
die Karriere anderer zu schädigen (geschlechtsspezifische Fürsorge drückt die Herausforderung), 
wird der geschlechtsspezifische Unterschied bei der Kommentierung zugunsten der Männer grösser sein, wenn die Zielperson stärker gefährdet ist. 
Tatsächlich sind die geschlechtsspezifischen Unterschiede in der Wahrscheinlichkeit, auf solche Artikel zu zielen, 
gering und statistisch nicht signifikant. Die Ergebnisse deuten also darauf hin, 
dass die unterschiedliche Risikoorientierung von Frauen an sich (sei es für sie selbst oder für die Zielpersonen) wahrscheinlich 
nicht der Grund für ihre geringere Rate an Kommentaren ist. 

Während sowohl Männer als auch Frauen eher auf Artikel abzielen, die zuerst von Männern verfasst wurden, 
ist dies bei Männern deutlich ausgeprägter. Da Frauen unter den Autoren von \gl{pnas}- und Science-Artikeln unterrepräsentiert sind, 
würde dies an sich schon die Gesamtzahl der Kommentare von Frauen im Vergleich zu Männern senken.
Es lässt sich nicht feststellen, ob diese Dynamik darauf zurückzuführen ist, 
dass Frauen andere Frauen als Zielpersonen bevorzugen (wie das Argument des geschlechtsspezifischen Risikos behauptet) 
oder dass Männer andere Männer überproportional bevorzugen (oder eine Kombination aus beidem). 
Es ist auch möglich, dass Frauen weniger als Männer auf Artikel abzielen, die von Männern geschrieben wurden, 
und zwar nicht aufgrund unterschiedlicher Konsequenzen, 
sondern weil ihr Fachwissen in Bezug auf die Arbeit von Frauen relevanter ist.
Dies weil Arbeiten, welche von Frauen verfasst wurden, auch ein wenig die Sichtweisen einer Frau repräsentieren.

Letztendlich konnte mit den Daten nicht direkt überprüft werden, ob Frauen mehr negative Konsequenzen erleiden, 
wenn sie den aktuellen Stand infrage stellen. Angesichts der aufgedeckten Muster und der mangelnden Unterstützung 
für die anderen Argumente bleibt dies jedoch eine plausible Erklärung für die niedrigere Rate an Kommentaren von Frauen.

% Conclusion
Die Studie kommt zu dem Ergebnis, dass sich Frauen seltener an akademischen Kommentaren beteiligen.
Eine Diskrepanz, welche die geschlechtsspezifischen Unterschiede bei der Veröffentlichung von Artikeln übersteigt. 
Diese Diskrepanz kann nicht durch Schwankungen im fachspezifischen Geschlechterverhältnis erklärt werden. 
Auch die Überrepräsentation von Frauen in den für Karriererisiken empfindlichsten Positionen oder 
eine grössere allgemeine Sensibilität für Risiken für andere können die Diskrepanz nicht erklären. 
Frauen richten auch einen geringeren Anteil ihrer Kommentare auf die Forschung von Männern als Männer. 
Zusammengenommen stimmen die beiden Ergebnisse am ehesten mit dem Argument überein, 
dass die geringere Anzahl von Kommentaren von Frauen auf die geschlechtsspezifischen Kosten zurückzuführen ist, 
die mit der Infragestellung massgeblicher Forschung verbunden sind, 
insbesondere wenn es um die Infragestellung traditioneller Statushierarchien geht. 
Dieses Argument kann aber mit den Daten nur indirekt überprüft werden.
 
Diese geschlechtsspezifischen Muster bei akademischen Kommentaren könnten den wissenschaftlichen Austausch 
zwischen Männern und Frauen behindern und Frauen innerhalb der wissenschaftlichen Gemeinschaft weiter marginalisieren. 
Wenn Expertinnen ausgeschlossen werden, fehlt es der akademischen Gemeinschaft als Ganzes an frischen Ideen und vielfältigen Meinungen.