\subsection{Methodik der Literaturanalyse}

% Intro Text Begründung weshalb
Um die Resultate der Arbeit in Kontext setzen zu können und Inspiration von existierender Forschung zu erhalten,
war es für uns - aber auch für den Erfolg der Arbeit - wichtig, den aktuellen Stand der Forschung aufzuarbeiten.
Da die Menge an zur Verfügung stehenden Informationen sehr gross ist, war ein methodischer Ansatz gefragt.
Mithilfe unserer Betreuerin, Prof. Dr. Mascha Kurpicz-Briki,  haben wir die nachfolgende Herangehensweise besprochen, 
um die wichtigste Literatur abdecken zu können.

% Grundsätzliches Vorgehen
Die erarbeitete Methodik sah vor, dass als Erstes die relevantesten Portale von wissenschaftlichen Arbeiten
identifiziert werden sollten. Als wir diese definiert hatten, suchten wir als Nächstes einen Filter, der die Arbeiten auf etwa 100
potenziell relevante Artikel eingrenzen sollte. Aus diesem Kandidatenset wollten wir im Anschluss die Abstracts
durchlesen, um mindestens fünf und maximal 20 vergleichbare Arbeiten zu finden. Diese galt es dann, als Ganzes
zu lesen und in den nachfolgenden Unterkapiteln der Literaturrecherche zusammenzufassen.

\subsubsection{Auswahl der Portale}
Für die Auswahl der Portale stützten wir uns auf die Empfehlungen unserer Betreuerin und ergänzten diese mit
einer weiteren, uns bekannten Plattform. Schlussendlich durchsuchten wir die folgenden drei online Datenbanken mit
wissenschaftlichen Artikeln.
\begin{itemize}
    \item ACM\footnote{https://dl.acm.org}
    \item IEEE\footnote{https://ieeexplore.ieee.org}
    \item PLOS\footnote{https://journals.plos.org}
\end{itemize}
ACM und IEEE sind bekannte Datenbanken für wissenschaftliche Arbeiten aus der Informatik. PLOS ergänzt diese Auswahl mit einer sozialwissenschaftlichen Perspektive.

\subsubsection{Filterung der Portale}
Der Filter sollte so eng gefasst sein, dass er möglichst viele irrelevante Arbeiten ausschliesst, gleichzeitig
aber auch offen genug sein, damit wir eine genügend grosse Resultatliste erhalten. Um die geeignete Abfrage zu finden,
mussten wir Variationen des Filters ausprobieren, bis wir mit der Grösse des Resultats zufrieden waren.

Schlussendlich sollte unser Filter alle Artikel bezüglich \gl{gendergap} oder Bias in den Medien ausgeben.
Dies indem er die Abstracts nach Stichwörtern wie \enquote{gender}, \enquote{media} und \enquote{equality}
durchsucht. Damit wollten wir möglichst viele Arbeiten zum Thema \gl{gendergap} in den Online-Medien
finden. Die nachfolgende Abbildung \ref{plos-query} zeigt das entsprechende Query
für die Plattform PLOS.

\begin{figure}[h]
    \begin{verbatim}
                    (abstract:gender) AND 
                    (
                        (abstract:media) OR 
                        (abstract:news)
                    ) AND 
                    (
                        (abstract:gap) OR 
                        (abstract:inequality) OR 
                        (abstract:equality) OR 
                        (abstract:bias)
                    ) AND NOT
                    (
                        (abstract:"social media") OR 
                        (abstract:pandemic) OR 
                        (abstract:covid)
                    )
    \end{verbatim}
    \caption{Query zur Suche der relevanten Literatur}
    \label{plos-query}
\end{figure}

Da die ursprüngliche Liste viele Artikel zu Social Media Themen oder der Covid Pandemie enthielt, haben wir diese explizit
ausgeschlossen.

\subsubsection{Manuelle Aussortierung anhand der Abstracts}
Die Anwendung dieses Filters resultierte in einer Liste von 115 Arbeiten, deren Abstracts wir im Anschluss durchlasen
und dazu jeweils drei Fragen beantworteten.

\begin{enumerate}
    \item Befasst sich der Artikel mit Geschlechter(un)gleichheit in den Medien?
    \item Sind die untersuchten Medien textbasiert?
    \item Wurde die Auswertung mithilfe von \gl{nlp} durchgeführt?
\end{enumerate}

Schlussendlich entschieden wir uns dazu, diejenigen Arbeiten zu verwenden, bei denen wir die ersten zwei Fragen mit
\enquote{Ja} beantworten konnten. Die dritte Frage hätte uns geholfen, die Liste weiter einzuschränken, wären
mit den ersten zwei noch zu viele Arbeiten übrig geblieben.

Bei der Auswertung der Liste haben wir definiert, was wir unter textbasierten Medien alles einschliessen wollen.
Wir entschieden uns dazu, Nachrichtenportale und wissenschaftliche Arbeiten als Medien zu definieren, Schulbücher
und Kinderbücher jedoch aus dieser Definition auszuschliessen.

Nach dieser manuellen Suche nach den relevantesten Arbeiten enthielt unsere Liste noch acht Artikel aus den Portalen PLOS und ACM.

\subsubsection{Analyse der Arbeiten}
Diese acht Arbeiten haben wir im Anschluss gelesen und in den nachfolgenden Kapiteln zusammengefasst. Der Fokus lag
dabei auf den Themen der Geschlechter(un)gleichheit und den Analysemethoden, die der unseren ähneln. Für diese Arbeit irrelevante
Themen, die ebenfalls in den Artikeln beschrieben wurden, haben wir weggelassen, um die Arbeit nicht unnötig zu verlängern.

\subsubsection{Abweichungen von der Methode}
In zwei Fällen sind wir von der oben beschriebenen Methodik abgewichen. Wir kannten aus der Vorarbeit noch einen
weiteren Forschungsartikel, der uns relevant für unsere Arbeit schien. Dieser war in der Liste jedoch nicht enthalten, da er aus einem
anderen Portal stammt. Es handelt sich dabei um die Arbeit namens \citetitle{gender_bias_in_media} \cite{gender_bias_in_media}.
Die Zusammenfassung dieser Arbeit finden Sie zusätzlich zu den anderen in den nachfolgenden Seiten. Die zweite Abweichung
ist der Ausschluss der Arbeit \citetitle{iet-minoritized-communities} \cite{iet-minoritized-communities}. Es hat sich herausgestellt,
dass dieser Artikel keine Untersuchung beschreibt, sondern die Zusammensetzung und Stossrichtung einer geplanten Forschungsgruppe.
Diese wird unter anderem auch die Geschlechter(un)gleicheit in den Medien untersuchen. Im vorliegenden Paper ist jedoch noch kein
Ergebnis vorhanden. Diese Tatsache war im Abstract jedoch nicht klar beschrieben.