% Er umfasst eine halbe bis eine Seite bzw. 150–250 Wörter.

% Worum geht es?
% Datenanalyse zum Bestimmen des Gender Gaps
% Basierend auf Daten des Vorprojekts
% Purpose?

Die Bachelor Thesis \enquote{Gender Gap Tracker für die Schweizer Medien} untersucht den \gl{gendergap}
in textbasierten deutschschweizer Online-Medien anhand der Anzahl Zitate von Männern und Frauen.
Die Datengrundlage bildet eine Datenbank mit über 350'000 Artikeln aus vier Nachrichtenportalen, die in einem separaten
Vorprojekt \cite{project2} mittels Webcrawling aufgebaut wurde.
Die Inspiration für dieses Projekt und das gewählte Vorgehen kam von der kanadischen Studie \citetitle{gender_gap_tracker}
\cite{gender_gap_tracker}.

% Wie bist du vorgegangen?
Zum Ermitteln des Gender Gaps setzt diese Arbeit - in Anlehnung an die Vorbildsstudie - auf \along{ml} basierte Werkzeuge aus dem Bereich
des \along{nlp}s. Die softwarebasierte Auswertung verwendet ein Sprachmodell von \gl{spacy}
um mithilfe von \ashort{ner}, \ashort{pos}, \gl{cr} und \gl{dependency-parsing} Zitate und Personen aus den
Texten zu extrahieren. Zum Überprüfen der Qualität dienen manuell erstellte Test Sets, anhand derer die
Performance der Software gemessen werden kann.

% Was sind deine wichtigsten Ergebnisse?
Die Resultate zeigen einen mittleren \gl{gendergap} von 50.34\%, wobei der kleinste bei 41.77\% und der grösste
bei 57.70\% liegt. Das bedeutet, dass der Algorithmus im Schnitt 50.34\% weniger Zitate von Frauen gefunden hat als von Männern.


% Was bedeuten deine Ergebnisse?
Unter Berücksichtigung der Limitationen (Kapitel \ref{limitations}) bedeuten diese Ergebnisse, dass in der Deutschschweiz ein
signifikanter \gl{gendergap} in den textbasierten Online-Medien vorliegt und Frauen deutlich unterrepräsentiert sind, 
gemessen an der Anzahl der gefundenen Zitate. Der Unterschied ist damit grösser, als in den meisten anderen Studien, 
die für diese Arbeit zum Vergleich herbeigezogen wurden (Kapitel \ref{state-of-the-art}).