% Die Diskussion besteht aus 4 Teilen:
% Zusammenfassung der Ergebnisse
% Interpretation der Ergebnisse
% Beschränkungen der Forschung
% Empfehlung für weiterführende Forschung

 

\section {Zusammenfassung der Ergebnisse}

% Zusammenfassung der Ergebnisse

% Zu Beginn deiner Diskussion gibst du nur eine kurze Zusammenfassung deiner 
% relevanten Ergebnisse. Du solltest hier auch deine Problemstellung erneut 
% darlegen. So kannst du deine Diskussion direkt im Bezug auf dein 
% Forschungsinteresse formulieren.

Das Resultat der durchgeführten Untersuchung zeigt einen klaren und erstaunlich
gleichmässigen \gl{gendergap} in allen analysierten Nachrichtenportalen von
41.77\% bis 57.70\%. Die Ergebnisse erfüllen damit die gestellte Aufgabenstellung, indem sie
die Grösse dieses Gaps in den grössten schweizer Gratis-Online-Nachrichtenportalen
benennen.

\section {Interpretation der Resultate}\label{interpretation}

% Interpretation von Ergebnissen

% Projekt von Ringier
% Original GGT
% Does Gender Matter in the News? Detecting and Examining Gender Bias in News Articles
% A Large-Scale Test of Gender Bias in the Media
% Bezug auf Resultate von anderen Studien
% Erklärung/Vermutung des Unterschieds

% Besprich die Forschungsergebnisse und gib an, inwiefern deine Erwartungen erfüllt 
% oder nicht erfüllt wurden. Diese Erläuterungen kannst du aus deiner verwendeten 
% Literatur ableiten, sie können jedoch auch auf deinen eigenen logischen Überlegungen 
% basieren. Auf jeden Fall solltest du beschreiben, wie deine Ergebnisse in den Rahmen
% passen, den du mittels Einleitung, Theorie und Forschungsfragen sowie Hypothesen 
% aufgestellt hast. Halte ebenfalls fest, inwiefern deine Resultate neue oder andere 
% Erkenntnisse in Bezug auf dein Thema mit sich bringen. Gehe alle Möglichkeiten durch: 
% Was genau hast du herausgefunden?

Die resultierenden Ergebnisse der Untersuchung ergänzen das Bild, das von der aktuellen Forschung
gezeichnet wird (vgl. Kapitel \ref{state-of-the-art}). Der \gl{genderbias} in den Medien wurde bereits auf unterschiedlichste
Arten gemessen: anhand der Anzahl Nennungen von Frauen und Männern \cite{does-gender-matter-in-the-news},
anhand von Zitaten \cite{gender_gap_tracker}, Quellenangaben \cite{citation-network} oder absolut
vergleichbaren Paaren \cite{gender_bias_in_media}. Stets scheint der gemessene Bias, der im Endeffekt
den Unterschied im Raum misst, der den Geschlechtern in den Medien eingeräumt wird, zwischen 20\%
und 60\% zu liegen, je nach Studie.

Im Vergleich zum kanadischen \gl{ggt} \cite{gender_gap_tracker}, der die Inspiration für diese Arbeit
geliefert hat, sind die Ergebnisse ähnlich aber im Schnitt doch etwas schlechter. So lag der \gl{gendergap}
über alle Nachrichtenportale in der Arbeit \citetitle{gender_gap_tracker} bei 42\%, bei unseren Resultaten jedoch bei
50.34\%. Dies widerlegt die Vermutung, die wir in der Vorarbeit \cite{project2} auf Basis des \citetitle{gggr-20}
\cite{gggr-20} getroffen haben. Wir sind davon ausgegangen, dass der \gl{gendergap} in der Deutschschweiz
etwas geringer ausfällt, als in Kanada, weil die Schweiz im internationalen Vergleich eine deutlich bessere
Geschlechtergerechtigkeit kennt, als Kanada \cite{gggr-20}.

Die Studie \citetitle{does-gender-matter-in-the-news} \cite{does-gender-matter-in-the-news} kommt in Bezug
auf den \gl{gendergap} je nach Datensatz auf sehr unterschiedliche Ergebnisse. So weisen die Artikel aus \gl{mind}
einen Gap von 53.9\% auf, gemessen an der Anzahl Nennungen von Männern und Frauen in Abstracts von Nachrichtenartikeln.
Diese Zahl ist vergleichbar, mit dem Resultat, das wir in dieser Auswertung gefunden haben.
\gl{ncd} schneidet mit 14.5\% unter Betrachtung dieser Metrik sehr gut ab. Doch zu beachten ist bei diesem Resultat die
Verteilung der Artikel über die Kategorien. Denn in den Kategorien \enquote{Style \& Beauty}, \enquote{Parenting} 
und \enquote{Entertainment} sind Frauen deutlich übervertreten. Trotzdem ist der \gl{gendergap} in diesem Datensatz
deutlich geringer, als derjenige, den wir gefunden haben.

Auch in der Studie \citetitle{gender_bias_in_media} \cite{gender_bias_in_media} wird ein starker Bias in der Anzahl
Nennungen von Personen der unterschiedlichen Geschlechter gemessen. So liegt der \gl{gendergap} im dritten Experiment,
das Paare von Frauen und Männern im gleichen Alter, in derselben beruflichen Kategorie von gleichem öffentlichen Interesse vergleicht
bei 46.71\%. Eine grosse Zahl, jedoch immer noch etwas kleiner als derjenige, den wir gemessen haben. So scheint der
\gl{gendergap} in der Deutschschweiz insgesamt grösser zu sein als im amerikanischen und kanadischen Raum.

Recherchen zu ähnlichen Projekten in der Schweiz ergeben jedoch ein hoffnungsvolles Bild. So misst der \gl{ringier} Verlag
in seinem Projekt \gl{equalvoice} \cite{ringier-equalvoice} die Repräsentation von Männern und
Frauen in seinen Medienhäusern anhand des \enquote{Body Counts}, der Anzahl Nennungen von Männern und Frauen und hat im
Jahr 2022 einen Unterschied von noch 34.95\% festgestellt. Wobei auch hier ein grosser Unterschied bei den unterschiedlichen Zeitungen
feststellbar ist. So schneidet die Print Ausgabe von Blick mit 42\% auch in dieser Auswertung ähnlich schlecht ab, wie
in den vorliegenden Resultaten mit 49.59\%. Es ist jedoch zu beachten, dass das Resultat von \gl{equalvoice} nur das Jahr 2022
betrifft und unsere Auswertung jedoch bis ins Jahr 2004 zurückreicht. Leider ist auf der Webseite auch zu erkennen, dass
sich der \gl{gendergap} seit dem Ende der Coronapandemie wieder weitet. \citeauthor{gender_gap_tracker} mutmassen in ihrer Arbeit,
dass dies auf den reduzierten Fokus auf dem von Frauen dominierten Gesundeheitsbereich zurückzuführen ist \cite{gender_gap_tracker}.
Es ist spannend, zu erkennen, dass diese Veränderung sowohl in Kanada wie auch in der Schweiz die gleichen Effekte zu haben scheinen.

Obwohl der sich der \gl{gendergap} wieder etwas geweitet hat, ist es jedoch höchst erfreulich, dass sich die Verlage zunehmend Gedanken zu diesem Thema machen und Lösungen anstreben.
So hat uns SRF bei unserer Anfrage nach einer Stellungnahme darauf hingewiesen, dass sie beispielsweise in der beliebten schweizer Radiosendung 
\enquote{Echo der Zeit} \footnote{https://www.srf.ch/audio/echo-der-zeit} im Februar erstmals mehr Gesprächspartnerinnen
als Gesprächspartner in den Echo-Gesprächen hatten \footnote{Dieser Beitrag konnte öffentlich nicht gefunden werden.}.
Es scheint als wären sich zumindest SRF und Ringier (Blick Mutterkonzern) der Problematik bewusst zu sein und Aufwand zur
Verringerung des \gl{gendergap}s zu betreiben. Trotzdem scheint die Deutschschweiz auf diesem Gebiet im Vergleich mit
den amerikanischen und kanadischen Nachrichtenportalen hinterherzuhinken und noch viel Arbeit betreiben zu müssen, um
den Unterschied weiter zu verkleinern.

Unter dieser Betrachtung stellt sich auch die Frage, was die Ursachen für dieses Problem sind.
So zeigt beispielsweise die Arbeit \citetitle{gender_bias_in_media} \cite{gender_bias_in_media} auf,
dass dieser Bias nicht alleine durch eine Präferenz der Medienschaffenden zur Porträtierung von Männern zustande kommt,
sondern auch der Struktur unserer Gesellschaft geschuldet ist, in der ein Grossteil der nachrichtenwerten
Positionen von Männern belegt wird. Dies ist nicht zuletzt dem Phänomen des \gl{glass-ceiling} geschuldet,
wie verschiedene Studien zeigen \cite{glass_ceiling_effect,glass_ceiling_politics}.

Die Bereinigung dieser Ungerechtigkeit kann also nicht von den Medienschaffenden alleine gelöst werden, denn
ihre Aufgabe in unserer Gesellschaft besteht in erster Linie in der Informationsverbreitung und der Schaffung
von Transparenz in den für die Öffentlichkeit interessanten Bereichen. Trotzdem können sie durch bewusste und reflektierte
Berichterstattungen und Recherchen ihren Anteil am Problem verringern und so die Gesamtsituation verbessern.
Die eigentliche Herkulesaufgabe, die Transformation unserer Gesellschaft zu einer ausgeglicheneren und
faireren Zivilisation, wird jedoch noch viel Effort von der Politik, Wirtschaft und jedem und jeder einzelnen von uns erfordern. % :)

\section {Limitationen}\label{limitations}

% Begrenzungen (limitations) deiner Untersuchung

% Die Begrenzungen bzw. Limitationen deiner Arbeit – im Englischen ‚limitations‘ 
% – erörterst du in einem eigenen Absatz innerhalb der Diskussion deiner Bachelorarbeit 
% bzw. Masterarbeit. Hier gehst du darauf ein, inwiefern du bei deiner Forschung an
% Grenzen gestoßen bist und welche Auswirkungen sich auf deine Ergebnisse ergeben haben.
% Wenn es ein paar Randbemerkungen zu deiner Forschung gibt oder du bestimmte
% Limitationen stark zu spüren bekommen hast, können diese auch eine Erklärung für deine
% Endresultate sein. Du kannst z. B. angeben, ob Fragen offen geblieben sind und auf
% Basis deiner Resultate Empfehlungen für zukünftige Forschung aussprechen. Achte 
% jedoch darauf, dass du deine eigene Forschung nicht gänzlich schlechtredest: Es ist
% nicht das Ziel, dass du eine Zusammenfassung aller kleinen Fehler erstellst. Über
% diese hättest du bereits nachdenken müssen, bevor du mit deiner Forschung angefangen
% hast.
Die Aufgabe, Zitate aus natürlichem Text zu extrahieren, ist eine schwierige. Die
menschliche Sprache kennt unzählige Wege, wie die Botschaft einer Person wiedergegeben
werden kann. Die Botschaft und die Tatsache, dass sie von einer anderen Person wiedergegeben wird,
ist durch die Syntax des Textes und den Kontext gegeben. \citeauthor{gender_gap_tracker} haben in
ihrer Arbeit \cite{gender_gap_tracker} zwei Arten von Zitaten identifiziert, für welche sie unterschiedliche
Herangehensweisen in ihrem Parsingprozess definierten. Doch auch sie konnten aufgrund der Regelmässigkeiten dieser
zwei Typen nicht alle Zitate herausfiltern und mussten auf heuristische Methoden zurückgreifen,
um übriggebliebene Zitate erkennen zu können. Die Struktur natürlichen Textes ist ambiguös und dessen Bedeutung
kontextabhängig. Obwohl gängige \gl{ml} gestützte
\gl{nlp}-Werkzeuge wie \gl{spacy} heute besser sind, diese Eigenschaften zu erkennen und einzuordnen,
scheinen die Resultate nur in einfachen Satztstrukturen zuverlässig zu funktionieren, wie unsere
Erfahrungen zeigten (vgl. Kapitel \ref{not-recognized-quotes}). So war es für uns in der kurzen Zeit
leider nicht möglich, den Prozess der Zitat-Extraktion zu verfeinern und um weitere Zitattypen zu erweitern.
Diese Einschränkung hat den grössten Einfluss auf die Qualität der vorliegenden Resultate.

Das bedeutet, dass wir schätzungsweise 60\% der gängigsten \textsl{Syntaktischen Zitate} gefunden haben. Diese Schätzung
basiert auf der Annahme, dass unsere Testresultate ein zuverlässiges Bild über die Performance auf den
restlichen Daten abgeben. Da es neben den \textsl{Syntaktischen Zitaten} auch noch die \textsl{Schwimmenden Zitate} gibt,
stellen diese nur einen Teil der Gesamtmenge der Zitate dar.
Ohne genauere Angaben zu der Verteilung der Anzahl Zitate auf deren
unterschiedlichen Kategorien zu kennen, schätzen wir den Anteil der \textsl{Syntaktischen Zitate} auf etwa 50\%.
Denn \textsl{Schwimmende Zitate} kommen unbegleitet von \textsl{Syntaktischen Zitaten} eigentlich nicht vor,
weil sonst der Kontext fehlen würde und die lesende Person keine Informationen zum Subjekt hätte.
Doch wenn sie vorkommen, treten sie meist in grösseren Mengen auf (bspw. in längeren Interviews).
Unter der Annahme, dass die verwendeten Tests ein repräsentatives Bild der Performance
des Algorithmus abgeben und dass die oben beschriebene Schätzung korrekt ist, hat das Programm
60\% von 50\%, also insgesamt 30\%, aller Zitate erkannt.
Es könnte also theoretisch sein, dass diejenigen 70\% der Zitate, die wir nicht erkannt haben, hauptsächlich von
Frauen sind. Dann hätten wir einfach "Pech" gehabt, dass der Algorithmus mit seiner Methode hauptsächlich Zitate
von Männern extrahiert, obwohl das Verhältnis effektiv anders wäre.
Aufgrund der grossen Datenmenge und der
Ergebnisse anderer Studien und Projekte glauben wir, dass dem nicht so ist und unsere Ergebnisse einen tatsächlichen
\gl{gendergap} in den deutschschweizer online-Medien widerspiegeln.
Doch mit Sicherheit lässt sich dies nicht sagen.

Die verwendeten Werkzeuge verfügen ihrerseits über Ungenauigkeiten. So gibt \gl{spacy} für das
von uns verwendete Model \enquote{Accuracy Scores} von > 90\% an \footnote{https://spacy.io/models/de\#de\_core\_news\_lg-accuracy}.
Für das \gl{pos} Tagging wird eine Präzision von 98\% versprochen und für das gelabelte 
\gl{dependency-parsing} eine von 91\%.
Auch bei der Erkennung der Personen herrscht eine gewisse Unsicherheit durch die
probabilistische Natur des \gl{spacy} Modells. So verkündet die offizielle Webseite zwar
eine Accuracy von 98\%, doch nach unserer Erfahrung ist zumindest die Anzahl von False-Positives
deutlich höher. Wir konnten beobachten, dass die Entitäten
zwar erkannt wurden, jedoch unter einem falschen Label. Meist waren dies Organisationen oder Orte,
die \gl{spacy} als Personen identifiziert hatte. Leider konnten wir nicht herausfinden,
welche Metrik \gl{spacy} für die Messung der Accuracy verwendet. Aufgrund der subjektiven
Erfahrung gehen wir aber davon aus, dass es sich wohl um eine Metrik handeln muss, die
False-Positives nicht oder nur schwach berücksichtigt. So könnte es sich bspw. um den \gl{recall}
handeln.

\section{Weiterführende Forschung}\label{further-research}

% Empfehlungen für weiterführende Forschungen

% Die Diskussion deiner Bachelorarbeit bzw. Masterarbeit kann mit einem Absatz zu 
% Empfehlungen für eventuell weiterführende Forschungen enden. Wie können andere 
% Forschende auf deiner Forschung aufbauen? Wo ergeben sich Ansatzpunkte für weitere 
% Untersuchungen? Vermeide Aussagen im Stil von ‚Es wird noch viel Forschung nötig sein‘. 
% Es ist nicht das Ziel, dass andere sich Ergänzungen zu deiner Arbeit überlegen müssen.
% Nenne ein paar konkrete Vorschläge für unabhängige weiterführende Untersuchungen,
% denen deine Arbeit als Orientierung dienen kann.

Die Resultate legen nahe, dass in den untersuchten Nachrichtenportalen ein grosser
\gl{gendergap} herrscht. Der zeitliche Rahmen hat jedoch weiterführende Untersuchungen
und Qualitätsverbesserungen verhindert. So bestünde der erste Schritt der weiterführenden Forschung
wohl darin, die Qualität der Ergebnisse zu verbessern und mit weiteren Tests sicherzustellen.
Dazu könnten weitere Testfälle manuell definiert werden für die \textsl{Syntaktischen Zitate}.

Ein weiterer wichtiger Punkt wäre die Extraktion der \textsl{Schwimmenden Zitate} mit entsprechenden
Test Sets. Erst wenn auch diese zu einem zufriedenstellenden Teil analysiert werden, kann eine abschliessende
Aussage zum \gl{gendergap} in den deutschschweizer online-Medien getroffen werden.

Weiterhin wäre es spannend zu erfahren, wie gross der Unterschied in den unterschiedlichen Kategorien von Nachrichtenartikeln ist.
Diese Erkenntnisse könnte man in Vergleich setzen
mit der Arbeit \citetitle{gender_bias_in_media} \cite{gender_bias_in_media} von \citeauthor{gender_bias_in_media} oder
dem Paper \citetitle{does-gender-matter-in-the-news} von \citeauthor{does-gender-matter-in-the-news}.
Des Weiteren wäre von Interesse, welcher Anteil der strukturelle Bias in unserer Gesellschaft
an diesem Gap hat und welcher Anteil dem Bias in der Berichterstattung von Medienschaffenden
geschuldet ist. Dazu könnten interessierte Forschende eine ähnliche Methodik wie
\citeauthor{gender_bias_in_media} in \citetitle{gender_bias_in_media} \cite{gender_bias_in_media}
anwenden.

Eher moralischer Natur ist die Forschungsfrage, ob dieser Gap überhaupt etwas
Schlimmes ist. Die Antwort zu dieser Frage liegt wohl in der Meinung der von
dieser Tatsache diskriminierten Gruppe, der Frauen. Hierzu könnte eine Meinungsumfrage
Aufschluss bieten.

Interessant wäre auch zu wissen, welche Auswirkungen sich aus diesem Gap ergeben.
Es ist unbestritten, dass Medien die Meinung einer Gesellschaft prägen. So kann
wohl auch angenommen werden, dass die Unterrepräsentation der Frauen in den Medien
Nachteile für diese mit sich bringt. Wie diese Nachteile aussehen, die sich aufgrund
dieser Unterrepräsentation ergeben, lässt sich aufgrund des aktuellen Forschungsstands
nicht sagen. Obwohl die strukturellen Benachteiligungen der Frauen gegenüber den Männern
klar benennbar sind, ist es schwierig diese auf einzelne Ursachen zurückzuführen. 
Weiterführende Forschung könnte hier Klarheit schaffen.
