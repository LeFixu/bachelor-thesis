\section{Persönliche Erkenntnisse}

\subsection{Literaturrecherche}
% Viel Hilfe von Frau Kurpicz zu Beginn der Phase beim Definieren der Methode
% Wir hatten genug vom Lesen und Zusammenfassen
% Wir waren ungeübt und mussten das Literatur-Analysieren erlernen (war schwierig)

Zu Beginn der ersten Phase unserer Arbeit, der Literaturrecherche, hatten wir das Glück, das Prof. Dr. Mascha Kurpicz-Briki uns bei der Definition unserer Forschungsmethode unterstützte. 
Sie half uns eine geeignete Methode für die Literaturrecherche zu entwickeln. 
Ihre Anleitung und Erfahrung waren sehr hilfreich und führten uns in eine richtige Richtung.

Während der Literaturrecherche hatten wir mit einer beträchtlichen Menge an wissenschaftlicher Literatur zu tun. 
Das Lesen und Zusammenfassen von zahlreichen Papers wurde mit der Zeit anstrengend und ermüdend. 
Es war eine Herausforderung, die Konzentration aufrechtzuerhalten und gleichzeitig den Überblick über die wichtigsten Informationen zu behalten.

Das Analysieren der gefundenen Literatur stellte eine neue und herausfordernde Aufgabe dar. 
Wir waren ungeübt in diesem Bereich und mussten uns mit den grundlegenden Techniken des Literatur-Analysierens vertraut machen. 
Es erforderte Zeit und Mühe, die verschiedenen Ansätze zur Bewertung und Zusammenfassung von Forschungsergebnissen zu erlernen. 
Es gab Momente, in denen es frustrierend war. Mit der Zeit und Übung konnten wir unsere Fähigkeiten verbessern.
Wir haben gelernt, dass Ausdauer und Organisation essenziell sind, um mit der Fülle an Informationen umzugehen. 

\subsection{Programmieren}
% Spass am programmieren
% Mehr Zeit als geplant für Date Bereinigung und Umgang mit dieser grossen Datenmenge (Skript Aggregation 40h)

Nach der Literaturrecherche haben wir festgestellt, dass es für uns wichtig ist, Spass am Programmieren zu haben und nicht ausschliesslich mit 
dem Lesen und Dokumentieren von Papers beschäftigt zu sein. 
Umso mehr hatten wir Freude aktiv zu programmieren und Lösungen zu entwickeln, die einen Mehrwert für unsere Arbeit bieten.
So konnten wir unsere Motivation und Kreativität aufrechterhalten, weil in der letzten Phase der Fokus dann wieder auf dem Dokumentieren liegt. 

Des Weiteren haben wir festgestellt, dass die Bearbeitung grosser Datenmengen mehr Zeit in Anspruch nehmen kann, als ursprünglich geplant. 
Insbesondere die Datenbereinigung erforderte viele Stunden bis alle Daten verarbeitet waren. 
Oftmals zwischen 20 und 30 Stunden pro Nachrichtenportal. 
Teilweise sogar mehrmals, da wir Fehler im Script bemerkten und diese nochmals neu starten mussten.
Bei der Aggregation waren auch 40 Stunden erforderlich, um alle Daten korrekt zu verarbeiten. 
Für zukünftige Projekte würden wir dies bei der Zeitplanung mehr berücksichtigen und ausreichend Zeit für die Datenbereinigung und Aggregation einplanen.

Insgesamt betrachtet war das Programmieren mit Python für uns der beste Teil der Arbeit, da wir immer wieder neue Herausforderungen und Lernmöglichkeiten entdeckten. 

\subsection{LaTeX}
% Vorkenntnisse mit LaTeX vom Projekt 2 waren Hilfreich

Unsere Vorkenntnisse in LaTeX aus Project 2 waren äusserst hilfreich. 
Sie haben uns ermöglicht, effizient mit unserem LaTeX-Dokument umzugehen und ansprechende Ergebnisse zu erzielen.

Dank unserer vorhandenen LaTeX-Kenntnisse konnten wir mit geübter Hand neue Kapitel erstellen, Formatierungen anpassen und Inhalte strukturieren. 
Wir waren mit den grundlegenden Befehlen und Konzepten vertraut, was uns geholfen hat, uns auf den Inhalt unseres 
Dokuments zu konzentrieren, anstatt viel Zeit mit dem Erlernen der grundlegenden Syntax zu verbringen wie im Vorprojekt.

\subsection{MongoDB}
Auch mit MongoDB hatten wir hilfreiche Erfahrungen aus dem Vorprojekt.
Die integrierte Abfragesprache \enquote{MongoDB Query Language} war uns auch schon bekannt und so konnten wir viel wirksamere Abfragen machen und
deren Komplexität steigern.

\subsection{Projektplanung}
% Es war schwiereig so detailiert zu planen im Voraus, aus dem Vorprojekt hatten wir nur auf Wochenbasis geplant.
% Zeitplan im grobben eingehalten
% Task von Datenbereinigung war am Anfang nicht fest eingeplannt und kahm noch dazu
% Sprint Board pro Projektphase half die Übersicht zu behalten vom Stand der Arbeiten
% Das Schreiben von Besprechungs Protokollen hat uns bei der Nachvollziehbarkeit von Entscheidungen geholfen und bei der Ausführung der Aufträge unterstützt

Im Vergleich zum Vorprojekt war es herausfordernd, die Planung im Voraus so detailliert durchzuführen. 
In diesem hatten wir hauptsächlich auf Wochenbasis geplant, doch während der Bachelor Thesis ist ein umfassender Projektplan Teil der Bewertung. 
Das Festlegen von konkreten Aufgaben und Meilensteinen auf langfristiger Basis erforderte eine sorgfältige Abwägung und eine genaue 
Einschätzung der Aufwände.

Trotz der Komplexität der Projektplanung konnten wir den Zeitplan im Grossen und Ganzen einhalten (vgl. Abbildung \ref{gantt}). 
Dies war ein Erfolg, da es uns ermöglichte, die Fortschritte des Projekts zu steuern und sicherzustellen, dass wir die gesetzten Ziele 
erreichen können. 
Die Disziplin und Organisation beider Teammitglieder trugen dazu bei, dass wir den Überblick über die Arbeitsauslastung behielten und  
unsere Deadlines einhielten.
Die Verwendung eines Sprint Boards (vgl. Abbildung \ref{sprint-board-screenshot}) für jede Projektphase erwies sich als äusserst hilfreich. Es ermöglichte uns den Fortschritt der Aufgaben zu überblicken. 
Das Sprint Board ermöglichte uns, Aufgaben zu visualisieren, den Status jedes Tasks zu verfolgen und Engpässe oder Verzögerungen frühzeitig zu erkennen. 
Dadurch konnten wir flexibel auf Änderungen reagieren und die Prioritäten entsprechend anpassen.

In unserer Projektplanung war die Datenbereinigung nicht als einzelner Task vorgesehen. 
Es stellte sich jedoch heraus, dass dieser Task erforderlich war, um mit den gesammelten Daten arbeiten zu können. 
Das Hinzufügen dieses zusätzlichen Tasks brachte eine gewisse Herausforderung mit sich, da wir unsere Zeitplanung und 
Ressourcen neu anpassen mussten, um diese Aufgabe bewältigen zu können.

Das Schreiben von Besprechungsprotokollen erwies sich als wertvoll für die Nachvollziehbarkeit von Entscheidungen und die Umsetzung von Tasks. 
Die Protokolle von den Meetings mit Prof. Dr. Mascha Kurpicz-Briki sowie von den Besprechungen mit uns beiden halfen uns, wichtige Informationen und Diskussionsergebnisse festzuhalten. 
Ausserdem war es hilfreich sicherzustellen, dass alle Teammitglieder über den Stand der Dinge informiert waren und die vereinbarten Aufgaben richtig ausgeführt wurden.

\subsection{Zusammenarbeit in der Gruppe}
% Eingespieltes Teamwork durch Projekte die wir schon vorher zusammengemacht haben
% Feedback von Frau Kurpicz kam rasch und Kommunikation mit Ihr war gut
% "Hit The Ground Running" --> Von Anfang an voll dran
% Gute Aufteilung der Arbeit in Arbeitstakte so das wir je nach Aufgaben parallel daran arbeiten konnten
% Gegenseitiges Review war effizient
% Schwierigkeiten im Team besprochen und Lösungen gefunden z.B. das wir nur Syntaktische Zitate extrahieren
% Klassische Geschlechterrollen in der Projekt Zusammenarbeit identifiziert und verändert so das wir uns abwechslen mit Präsentatieren und der Meeting-Leitung
% Meetings mit Dozentin immer recht speditiv, zeitlicher rahmen eingehalten und meistens unter den abgaemachten 30min, regelmässiger Austausch
% Spontane Terminfindung hat nicht immer funktioniert --> Lösung: Regelmässiger Termin (ab Phase Praxisteil)

Bei unserer Selbstreflexion zur Zusammenarbeit in der Gruppe haben wir folgende wichtige Erkenntnisse gewonnen:

\begin{itemize}
    \item \textbf{Eingespieltes Teamwork durch vorherige Projekte:} Dank unserer vorherigen Zusammenarbeit in anderen Projekten konnten wir als Team gut zusammenarbeiten. Wir kannten bereits unsere Stärken und Schwächen und wussten, wie wir effektiv kommunizieren und unsere Aufgaben koordinieren können.
    \item \textbf{Schnelles Feedback von Prof. Dr. Mascha Kurpicz-Briki und gute Kommunikation:}  Das rasche Feedback unserer Betreuerin auf unsere Mails hat uns sehr geholfen. Die Kommunikation mit ihr war reibungslos und wir konnten jederzeit Fragen stellen und Unterstützung erhalten und haben diese Möglichkeit auch, wo nötig, in Anspruch genommen.
    \item \textbf{Hit The Ground Running:} Wir waren von Anfang an äusserst engagiert und haben sofort mit der Arbeit begonnen. Dadurch konnten wir effizient voranschreiten und die Projektziele frühzeitig erreichen.
    \item \textbf{Gute Aufteilung der Arbeit in Arbeitspakete:} Durch die sorgfältige Aufteilung der Arbeit in Arbeitspakete konnten wir parallel an verschiedenen Aufgaben arbeiten. Dies erhöhte unser Arbeitstempo.
    \item \textbf{Nützliche gegenseitige Reviews:} Das gegenseitige Review unserer Arbeit war sehr sinnvoll. Durch konstruktives Feedback konnten wir unsere Ergebnisse verbessern und sicherstellen, dass sie unseren Anforderungen entsprachen.
    \item \textbf{Umgang mit Schwierigkeiten im Team:} Wir haben Schwierigkeiten im Team offen angesprochen und gemeinsam Lösungen gefunden. Ein Beispiel dafür war die Entscheidung, nur \textsl{Syntaktische Zitate} zu extrahieren. Durch solche Diskussionen konnten wir unsere Arbeitsweise optimieren und bessere Ergebnisse erzielen.
    \item \textbf{Veränderung klassischer Geschlechterrollen:} Wir haben klassische Geschlechterrollen in der Projektzusammenarbeit identifiziert und aktiv verändert. Wir haben uns abgewechselt bei Präsentationen und der Leitung von Meetings, um sicherzustellen, dass alle Mitglieder des Teams gleiche Chancen haben, ihre Fähigkeiten zu entwickeln und zu zeigen.
    \item \textbf{Effiziente und zeitlich gut geplante Meetings mit Prof. Dr. Mascha Kurpicz-Briki:} Die Meetings mit unserer Betreuerin verliefen stets speditiv und innerhalb des festgelegten Zeitrahmens. Wir hatten regelmässige Austausche, bei denen wir den Fortschritt präsentierten, die weiteren Schritte besprachen und Fragen diskutierten.
    \item \textbf{Lösung für Terminfindung:} Die spontane Terminfindung hat nicht immer reibungslos funktioniert in unserem 2er Team. Als Lösung haben wir uns auf wöchentliche Termine ab der Praxisteil-Phase geeinigt, um sicherzustellen, dass wir uns regelmässig treffen. Um den Fortschritt unsrer Arbeiten kontinuierlich besprechen und uns über Probleme austauschen zu können.
    % \item \textbf{1. weiterer Kritik Punkt:} Stress und hohe Anforderungen, hohe Erwartungshaltung
    % \item \textbf{2. weiterer Kritik Punkt:} Startschwierigkeiten im Team
\end{itemize}

Insgesamt war die Zusammenarbeit in der Gruppe sehr positiv. 
Unsere vorherige Erfahrung als Team, effektive Kommunikation, gegenseitiges Feedback und die Fähigkeit, Schwierigkeiten 
anzusprechen und Lösungen zu finden, haben dazu beigetragen, dass wir so gut zusammen arbeiten konnten. 
Die identifizierten Verbesserungen in der Geschlechterrolle beim Meeting führen und die regelmässige Terminfindung haben ebenfalls 
zu einer positiven Teamdynamik beigetragen.