% Das Fazit steht im direkten Zusammenhang mit der Einleitung, 
% da du auf die Forschungsfragen oder Hypothesen eingehst, 
% die zu Beginn der Bachelorarbeit aufgestellt wurden.

% Im Fazit deiner Bachelorarbeit wiederholst du nicht noch einmal die gesamte Arbeit,
% sondern du legst zusammenfassend die wichtigsten und aussagekräftigsten Ergebnisse dar. 
% Beim Schreiben des Fazits solltest du also die folgenden Regeln beachten:

%     Keine neuen Informationen und Interpretationen
%     Das Fazit ist grundsätzlich eine Zusammenfassung deiner wichtigsten Erkenntnisse. 
%     Es darf daher nur mit Informationen und Gedankengängen gefüllt werden, die du bereits im Fließtext dargelegt hast.
%     Keine Beispiele und Zitate
%     Im Fazit fasst du Fakten zusammen und erklärst sie nicht anhand neuer Beispiele und Zitate anderer Forschenden.
%     Dein Ergebnis ist immer wertvoll
%     Es kann vorkommen, dass deine Ergebnisse nicht deinen Erwartungen entsprechen. 
%     Wenn du deine Forschungsfrage aber gut gestellt hast, bspw. mit den Formulierungen‚
%     wie viel‘ oder ‚inwiefern‘, wirst du immer ein wertvolles Ergebnis erhalten, 
%     das die Forschung auf diesem Gebiet weiterbringt.


% Ziele der Arbeit (aus der Einleitung):
% Die Arbeit soll aufzeigen, inwiefern das Erkennen von Zitaten
% mittels \gl{nlp} Werkzeugen auch in deutschsprachigen Texten möglich ist. Anhand dessen
% soll sie analysieren, wie viele der gefundenen Zitate von Männern sind und wie viele von Frauen.

\section{Evaluation der Ziele}

Die vorliegende Arbeit hatte zwei grosse Ziele. Beide sind zuvor noch nie umgesetzt worden,
sofern wir das ermitteln konnten.

\begin{itemize}
    \item Die Arbeit sollte aufzeigen, inwiefern das Erkennen von Zitaten mittels \gl{nlp} Werkzeugen auch in deutschsprachigen Texten möglich ist.
    \item Sie sollte den \gl{gendergap} anhand der Anzahl Zitate von Männern und Frauen in den gesammelten Texten bestimmen
    und diesen mit den verwandten Arbeiten vergleichen.
\end{itemize}

Aus diesen zwei Hauptzielen ergaben sich Zwischenziele, die wir ebenfalls festhalten möchten.
Das erreichen dieser Zwischenschritte war essenziell für das Bestimmen des \gl{gendergap}s.
\begin{itemize}
    \item Der Algorithmus zum Bestimmen des Gender Gaps muss Personen und dazugehörige
    Informationen aus den Artikel Texten extrahieren können.
    \item Die Software soll in der Lage sein, das Geschlecht der identifizierten Personen zu bestimmen.
\end{itemize}

\subsection{Erkennen der Zitate}

Mithilfe des erstellten Programms konnten wir zeigen, dass das Erkennen von Zitaten mithilfe von \gl{nlp} Werkzeugen
auch in deutschschweizer online-Artikeln möglich ist. So konnte der Algorithmus in einem Datensatz
mit 351'021 einzigartigen Artikeln 133'443 Zitate extrahieren.

Innerhalb von zwei Wochen war es uns
möglich, schätzungsweise 60\% aller \textsl{Syntaktischen Zitate} zu extrahieren.
Wir sind überzeugt, dass mithilfe eines grösseren Testsets und Verfeinerungen des Algorithmus
eine höhere Präzision und Verlässlichkeit der Extraktion der \textsl{Syntaktischen Zitate} möglich ist.
Darauf aufbauend sollte auch die Erkennung der \textsl{Schwimmenden Zitate} möglich sein, vermuten wir.
Um eine verlässliche Aussage bezüglich des \gl{gendergap}s treffen zu können, müssten diese Verbesserungen und Erweiterungen unbedingt durchgeführt werden.

\subsection{Ermitteln des Gender Gaps}

Das Resultat der durchgeführten Analyse zeigt einen \gl{gendergap} von 50.34\% über alle Zitate auf.
Der grösste gemessene Gap pro Nachrichtenportal liegt bei 57.7\%, der kleinste bei 41.77\%.
Die Datengrundlage für diese Auswertung bildet ein bereinigtes Datenset von 351'021 einzigartigen
deutschsprachigen online-Artikel aus vier Nachrichtenportalen der Deutschschweiz. Die Analyse umfasst
schätzungsweise 60\% aller \textsl{Syntaktischen Zitate} und 30\% aller Zitate insgesamt.

Die gemessenen Unterschiede in der Anzahl Zitate lassen sich mithilfe der von uns definierten
\gl{gendergap} Formel (vgl. Abbildung \ref{ggt-formula}) mit anderen Arbeiten vergleichen.
Der direkteste Vergleich lässt sich mit dem \enquote{Body Count} von Blick aus der \gl{equalvoice}
Initiative anstellen. Die Print Ausgabe misst dabei einen \gl{gendergap} von 42\%, während unsere
Daten für das online Portal einen Gap von 49.59\% anzeigen.
Die Vorbildsstudie \citetitle{gender_gap_tracker} \cite{gender_gap_tracker} misst in den kanadischen Medien
einen \gl{gendergap} von 42\%.
\citetitle{does-gender-matter-in-the-news} \cite{does-gender-matter-in-the-news}
misst je nach Datensatz einen \gl{gendergap} von 53.9\% oder 14.5\%.
\citetitle{gender_bias_in_media} \cite{gender_bias_in_media} kann einen \gl{gendergap} von 46.71\% nachweisen.

Die berechneten \gl{gendergap}s der verwandten Arbeiten zeichnen ein ähnliches Bild, wie jenes
der vorliegenden Resultate. Es scheint als würde sich der Unterschied im Raum, der Männern
und Frauen zugesprochen wird - mit Ausnahmen natürlich - im Bereich von etwa 40\% - 50\% bewegen.

\subsection{Extraktion der Personen}
Die Kombination von \gl{ner} und \gl{cr} ermöglichte eine weitgehende Extraktion und Identifizierung 
der im Text genannten Personen. 
Durch den Einsatz dieser beiden Verfahren konnten die Personen grundsätzlich erfolgreich aus dem Text herausgefiltert werden.

Dennoch ist es wichtig zu betonen, dass die Extraktion von Personen aus Texten keine absolute Gewährleistung für eine fehlerfreie Identifizierung bietet. 
Es besteht immer die Möglichkeit von Fehlern, insbesondere wenn die Namen mehrdeutig sind, wenn im 
Text Unklarheiten auftreten oder Verweise auf nicht eindeutige Personen gemacht werden.

\subsection{Bestimmen des Geschlechts einer Person}
Das Geschlecht konnte bei 87.2\% aller gefundenen Personen, denen ein Zitat zugewiesen wurde, erfolgreich bestimmt werden. 
Die restlichen 12.8\% der Personen, bei denen das Geschlecht nicht ermittelt werden konnte, sind entweder vermutlich keine 
tatsächlichen Personen, da sie fälschlicherweise vom \gl{ner} als solche erkannt wurden, oder es handelt sich um Namen, 
denen kein eindeutiges männliches oder weibliches Geschlecht zugeordnet werden können.

Wir hatten auch die Idee, das Geschlecht der Autorinnen und Autoren der Artikel zu bestimmen. 
Leider war die Datenqualität der vorliegenden Datengrundlage aus dem Vorgängerprojekt unzureichend. 
Oftmals waren im Feld \enquote{author} der Datenbank keine Namen, sondern nur unbrauchbare Daten enthalten.
Wir vermuten, dass Teile des Artikels als Autor:in erkannt wurden, 
da die Autorinnen und Autoren Informationen häufig am Ende des Texts in einer nicht eindeutigen Form aufgeführt wurden.
