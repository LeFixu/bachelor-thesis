% First section: context

% 1971 - Ja zum Frauenstimm- und -wahlrecht (65\% dafür bei Stimmbeteiligung von 57.72\%)
Der siebte Februar 1971 war ein besonderer Tag für die Schweiz. Insbesondere für
die Schweizerinnen, denn an diesem Tag nahmen die Schweizer - im zweiten Anlauf nach 1959 - das
nationale Frauenstimmrecht mit 65\% bei einer Stimmbeteiligung von 57.72\% an
\cite{frauenstimmrecht}.

Ab diesem Zeitpunkt durften Frauen in der Schweiz erstmals wählen, abstimmen und sich zur Wahl
aufstellen lassen. Dieser Meilenstein befeuerte die Frauen in ihrem Streben nach Gleichberechtigung.
So wurden in den folgenden Jahren regelmässig Abstimmungen zur Annäherung der Rechte der Frau
an die des Mannes verabschiedet.

% 1981 Gleichstellung von Frau und Mann --> 1996 Das Gesetzt sorgt für ihre rechtliche und tatsächliche Gleichstellung
Der nächste grosse Schritt erfolgte 10 Jahre später, als die rechtliche Gleichstellung von Frauen und
Männern erstmals in die Verfassung aufgenommen wurde. Dieser Artikel wurde im Jahr 1996
mit der Ergänzung verschärft, dass der Bund nicht nur für die rechtliche, sondern auch für die
tatsächliche Gleichstellung der Frau sorgen soll
\cite{gleichstellung}.

% 1988 Neues Eherecht: Seit 1912 ist der Ehemann das Haupt der Gemeinschaft. Ab 1988 Gleichberechtigte Partnerschaft und gemeinsame Verantwortung der Ehegatten für die Pflege und Erziehung der Kinder und den Familienunterhalt
Ab 1988 stellte die Verfassungsreform zum Eherecht die Frau und den Mann in der Ehe gleich und
verpflichtete beide Parteien zur Pflege und Erziehung der Kinder und den Familienunterhalt
\cite{eherecht}.

% 1990 Appenzell Innerhoden wird gezwungen, das Frauenstimmrecht einzuführen. 28. April 1991 können die Frauen der Landsgemeinde zum ersten Mal politisch teilnehmen.
Im Jahre 1990 erst, fast 20 Jahre nach der nationalen Abstimmung, wurden auch die Frauen aus dem Kanton Appenzell Innerrhoden stimmberechtigt, als
das Bundesgericht den Kanton durch eine Neuinterpretation seiner Verfassung gezwungen hatte,
das Frauenstimmrecht einzuführen
\cite{frauenstimmrecht2}.
Trotz dieser gesetzlichen und gesellschaftlichen Fortschritte war die Schweiz im Jahre 1991 noch
weit entfernt von rechtlicher und tatsächlicher Gleichstellung der Geschlechter. 
Deshalb riefen 1991 Frauenorganisationen zum ersten nationalen Frauenstreik der Schweiz auf.

% 1991 1. Landesweiter Frauenstreik: halbe Million Frauen für "Gleiche Rechte für Mann und Frau"
Mehr als eine halbe Million Frauen gingen auf die Strasse im grössten Protest, den die Schweiz
seit dem Landesstreik 1918 gesehen hatte. Sie forderten die tatsächliche (nicht bloss rechtliche) Gleichstellung in der Gesellschaft,
die Bekämpfung der horrenden Lohnungleichheit am Arbeitsplatz, gleiche Ausbildung für Frauen
und ein Stopp der alltäglichen sexuellen Belästigung am Arbeitsplatz
\cite{frauensteik}.

% 1996 Gleichstellungsgesetzt tritt in Kraft --> Förderung tatsächlicher Gleichstellung: Diskriminierungsverbot bei der Erwerbsareit (Anstellung, Aufgabenzuteilung, Arbeitsbedingungen, Entlöhnun, Aus- und Weiterbildung, Beförderung und Entlassung). Sexuelle Belästigung gilt als Diskriminierung (erst ab 1996...)
Der Streik hatte auch einen grossen Einfluss darauf, dass 1996 schliesslich
die tatsächliche Gleichstellung der Frau gegenüber dem Mann in der Bundesverfassung
verankert wurde
\cite{gleichstellungsgesetz}.

% 2002 Abtreibungen werden erlaubt
Der nächste Meilenstein wurde 2002 mit der Legalisierung des Schwangerschaftsabbruchs (Abtreibung)
bis zur 12. Schwangerschaftswoche gesetzt
\cite{schwangerschaftsabbruch}.
% 2004 Gewalt in Ehe und Partnerschaft wird Offizialdelikt (Körperliche Gewalt, sexuelle Nötigung und Vergewaltigung in der Ehe oder Lebensgemeinschaft werden neu von Amtes wegen verfolgt)
2004 wurde ein weiteres zentrales Anliegen der Frauenrechtsbewegung adressiert: Gewalt
in der Ehe und in der Partnerschaft wurde als Offizialdelikt eingestuft. Dazu gehört Körperverletzung
wie auch sexuelle Nötigung oder Vergewaltigung
\cite{gewalt_frauen}.
In den darauffolgenden Jahren wurden Gleichstellungsfragen breit in den öffentlichen und
bald auch in den sozialen Medien diskutiert, dessen Höhepunkt sich wahrscheinlich in der
\#MeToo Debatte fand
\cite{metoo}.

% 2019 Zweiter Nationaler Frauenstreik: Lohngleichheit, Vereinbarkeit Familie und Beruf, Geschlechterstereotypen, Politische und Gesellschaftliche Teilhabe, Gewalt an Frauen, Anerkennung für Betreuungsarbeit
Diese Debatte feuerte 2019 auch den zweiten nationalen Frauenstreik an. Auch dieses Mal mobilisierten
die Frauen über eine halbe Million Demonstrierende auf ihren Märschen durch die schweizer Innenstädte.
Sie prangerten den immer noch anhaltenden Sexismus in der schweizerischen Gesellschaft und
das Problem der sexuellen Gewalt an und forderten Anerkennung für Betreuungsarbeit und die Eliminierung
der nach wie vor grossen Lohnungleichheit
\cite{wiki_frauensteik}.

% Vielleicht kurzer Abschnitt zum Thema "Gap"?

% Beschreibung der aktuellen Probleme

% Pay Gap
Denn zu diesem letzten Thema, der Lohnungleichheit oder dem sogenannten \gl{paygap}, hatte das Bundesamt für Statistik (BFS) % Bibtex funktioniert nicht richtig
nur einige Monate zuvor eine neue Studie veröffentlicht \cite{bfs-paygap}. Diese weist einen durchschnittlichen
Lohnunterschied von 18.3\% zwischen Männern und Frauen nach. Auch nach der Eliminierung von Faktoren wie
Ausbildung, Arbeitserfahrung oder Berufsgattung verbleiben 7.7\% unerklärlich, die auf einen
\gl{genderbias} beim Lohn hindeuten.

% Representational Gap
% Politk
Doch nicht nur beim Lohn gibt es immer noch bedeutende Unterschiede zwischen den Geschlechtern.
Frauen sind in unserer Gesellschaft in wichtigen Positionen der Politik und Wirtschaft stark
untervertreten. Nach den letzten Wahlen sassen im Nationalrat mit 96 Vertreterinnen gegenüber 
150 Vertretern zwar so viele Frauen in der Grossen Kammer wie noch nie, doch von der Hälfte sind
wir noch immer ein gutes Stück entfernt. Immerhin zeigt der Trend in die richtige Richtung mit einem
Plus von 14\% (+12) Frauen gegenüber dem Vorjahr
\cite{frauen_politik}.
Doch der konservativere Ständerat und die Kantonsräte hinken mit 26\% (12 / 46) respektive 30\% deutlich hinterher
\cite{frauen_nationalrat}.
% Wirtschaft
Noch gravierender ist die Situation in der Wirtschaft bei den Führungspositionen und in den Unternehmensleitungen. 
So ist der Frauenanteil in diesen Positionen im Verlauf von 2012 bis 2022 nur um knapp 2\%
auf 36\% angestiegen
\cite{frauen_statistiken_fpos}.


% Medien -> Überleitung
Neben den \gl{paygap} und dem sogenannten \gl{gender-bias-distribution} in der Politik und Wirtschaft
gibt es auch einen \gl{gendergap} in den Medien, wie verschiedene Studien beweisen
\cite{gender_gap_tracker,does-gender-matter-in-the-news,gender_bias_in_media}.
% Gender Gap in anderen Wissenschaftliche Arbeiten
% does gender matter in the news?
So messen beispielsweise \citeauthor{does-gender-matter-in-the-news} in ihrer Studie
\citetitle{does-gender-matter-in-the-news} \cite{does-gender-matter-in-the-news} in
zwei unterschiedlichen Datensets mit mit 160'000 bzw. 210'000 englischsprachigen
Nachrichtenartikeln einen signifikanten \gl{gender-bias-distribution} von bis zu 53.9\%.

% A large-scale test of gender bias in the media
Die Arbeit \citetitle{gender_bias_in_media} von \citeauthor{gender_bias_in_media} \cite{gender_bias_in_media} misst nicht nur
den \gl{gender-bias-distribution}, sondern teilt diesen zusätzlich auch auf in einen Bias,
welcher der Struktur unserer Gesellschaft mit überproportional mehr Männern in nachrichtenwerten
Positionen geschuldet ist und in einen zweiten Bias, der die Medienschaffenden dazu veranlasst,
mehr über Männer als Frauen zu schreiben.

% GGT
Aber die wichtigste Studie für diese Arbeit wurde von \citeauthor{gender_gap_tracker} unter
dem Titel \citetitle{gender_gap_tracker} \cite{gender_gap_tracker} veröffentlicht.
Sie untersucht den \gl{gendergap} in ausgewählten englisch- wie auch französischsprachigen
kanadischen Medien anhand moderner \gl{nlp} Methoden und schaltet die neusten Erkenntnisse
regelmässig online \cite{gender_gap_tracker-website} auf, damit Interessierte den Fortschritt (oder Rückschritt) auf täglicher Basis
mitverfolgen können. Zum Zeitpunkt der Veröffentlichung des Papers waren 71\% aller Zitate
Männern zuzuordnen.

% Second section: project
% Beschreibung der Ziele des Projekts
An diese Studie wollen wir mit unserer Arbeit anknüpfen und nach dem gleichen Prinzip
den \gl{gendergap} in den deutschschweizer Online-Medien messen. Zumindest in diesen vier,
von denen wir über Daten verfügen. Die Arbeit soll aufzeigen, inwiefern das Erkennen von Zitaten
mittels \gl{nlp} Werkzeugen auch in deutschsprachigen Texten möglich ist. Anhand dessen
soll sie analysieren, wie viele der gefundenen Zitate von Männern sind und wie viele von Frauen.
Dazu wollen wir zeitgemässe Werkzeuge des \gl{nlp} einsetzen, wie \gl{ner}, \gl{pos} Tagging
oder \gl{dependency-parsing} um bestmögliche Resultate zu erzielen. Diese fungieren als Grundlage
für unsere selbst geschriebenen Algorithmen, die je nach Aufgabe Personen herauslesen, Zitate
extrahieren oder das Geschlecht einer Person bestimmen.
Diese Resultate wollen wir im Anschluss zwischen den einzelnen Nachrichtenportalen vergleichen und mithilfe der \gl{gendergap} 
Formel (vgl. Abbildung \ref{ggt-formula}) in Kontext mit den vergleichbaren Arbeiten setzen.

Wir wünschen viel Spass beim Lesen!