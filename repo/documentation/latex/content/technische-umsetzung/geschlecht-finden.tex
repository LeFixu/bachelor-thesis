\section{Geschlecht identifizieren}\label{identify-gender}

% Liste Vornamen Bund 2021
% Pronemen Listen
% Genderzie API

Um das Geschlecht einer Person herauszufinden, werden als Erstes die offiziellen Listen 
\footnote{https://www.bfs.admin.ch/bfs/de/home/statistiken/bevoelkerung/geburten-todesfaelle/namen-schweiz.html} 
von Vornamen aus dem Jahr 2021 des Bundesamts für Statistik konsultiert. 
Davon gibt es zwei, eine mit männlichen und eine mit weiblichen Vornamen.
Dann werden die Pronomen der Person, falls vorhanden, verglichen, ob sie wahrscheinlicher männlich oder weiblich sind.
Wenn die Wahrscheinlichkeit aus den Pronomen und der Liste vom Bundesamts für Statistik unter 66\% liegt, dass es zum einten Geschlecht gehört,
wird die API von Genderize.io mit dem Vornamen abgefragt. Bei dieser API gibt es jedoch eine Begrenzung von 1000 Requests pro Tag.
Wenn dieses Limit erreicht ist, wird trotzdem das wahrscheinlichere Geschlecht aus dem Resultat der Liste vom Bundesamts für Statistik und den Pronomen gewählt.
Das Geschlecht der Person wird dann in der Form eines \textsl{Gender} Enums (vgl. Abbildung \ref{genderized-person-class}) zurückgegeben.

\begin{figure}[H]
	\begin{lstlisting}[language=Python]
    class GenderizedPerson(Person):
        gender: Optional[Gender]  # Unknown => None
        probability: float
    
    class Gender(Enum):
        UNKNOWN = 0
        FEMALE = 1
        MALE = 2        
	\end{lstlisting}
	\caption{GenderizedPerson Klasse und Gender Enum}
	\label{genderized-person-class}
\end{figure}
