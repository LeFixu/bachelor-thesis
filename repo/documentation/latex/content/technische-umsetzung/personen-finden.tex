\newpage
\section{Personen extrahieren}\label{people-extraction}

% 1. NER Entitäten mit Personen Tag aus Text lesen
% 2. Mit Coreference resolution Personen Cluster bekommen
% 3. NER Coreference und Coref-Res Entitäten verbinden um eindeutige Personen zu bekommen mithilfe der Schnittmenge der Wörter
% Dabei wird aus den Coreference Einträge neben Vornamen und Nachnamen der Person auch noch Pronomen und Artikel herausgelesen mit POS und der Person zugewiesen.
% diese sind dann zusätzlich zum Vornamen hilfreich für das Geschlecht der Person zu bestimmen.

Das Extrahieren von Personen aus einem Text umfasst mehrere Schritte.
Zunächst werden die Entitäten mit dem Tag \enquote{Person} mithilfe von \gl{ner} aus dem Text herausgelesen.
Als Nächstes wird \gl{cr} verwendet, um Personencluster zu bilden. 
Dies bedeutet, dass der Text auf Referenzen zu Personen überprüft wird, z.B. wenn eine Person im Text als \enquote{sie} oder \enquote{er} bezeichnet wird. 
\gl{cr} hilft dabei, alle Referenzen auf dieselbe Person zu identifizieren und in Gruppen zu organisieren.
Um eindeutige Personen zu erhalten, müssen die NER-Entitäten und die Coreference-Entitäten zusammengeführt werden.

Dazu werden die Schnittmengen der Wörter verwendet, die den NER- und Coreference-Entitäten zugeordnet sind.
Auf diese Weise können Personen und zugehörige Informationen eindeutig identifiziert werden.
Zusätzlich zu den Vornamen und Nachnamen der Person werden auch Pronomen und Artikel, die mit der Person in Verbindung stehen aus der Coreference-Entität, mithilfe von \gl{pos} 
extrahiert und der Person der pronouns\_and\_articles Liste zugewiesen. 
Damit die Zuordnung von den Zitaten zu den Personen im nächsten Schritt einfacher ist, speichern wir die Positionen im Text wo die Personen genannt wurde auch ab unter dem positions\_in\_text Attribut der Person Klasse.
Weil \gl{ner} Entitäten zusätzlich zum Namen auch Titel wie \enquote{Prinz} oder \enquote{Herzogin} haben können, 
wurden diese mithilfe einer eigens definierten Liste erkannt und unter dem substitute\_nouns Attribut abgelegt. 
All das wird in der Funktion \_\_get\_people\_from\_ner\_coref (vgl. Abbildung\ref{get-people-from-ner-coref}) gemacht, welche als Input eine Liste der \gl{ner} und \gl{cr} Entitäten bekommt.
Die zusätzlichen Informationen, welche der Algorithmus mittels der Coreference-Entität zur Person erhält, sind im nächsten Schritt hilfreich, um das Geschlecht der Person zu bestimmen.
Der Rückgabewert der Funktion ist eine Liste von \textsl{Person} Objekten. Diese Klasse ist in der Abbildung \ref{person-class} ersichtlich.

\begin{figure}[H]
    \begin{lstlisting}[language=Python]
        def __get_people_from_ner_coref(
            ner_people_list: List[Tuple[List[str], List[int]]], 
            coref_list: List[str]
        ) -> List[Person]:
            """
            This method creates people with the help of the NER list and the coref list
            """
            people = []
        
            for person_words in ner_people_list:
                new_person = Person("", "", [], [], [], [])
                for person_word in person_words[0]:
                    for coref_entry in coref_list:
                        if person_word.lower() in coref_entry:
                            pos_res = nlp(coref_entry)
                            for word in pos_res:
                                if word.pos_ == "PRON":
                                    new_person.pronouns_and_articles.append(word.text)
                                if word.pos_ == "DET":
                                    new_person.pronouns_and_articles.append(word.text)
                                if word.pos_ == "NOUN":
                                    new_person.substitute_nouns.append(word.text)
        
                # handle substitute nouns
                # z.B. wenn NER Eintrag Prinz Harry ist, dann wird Prinz zu substitute_nouns Liste der Person hinzugefügt
                person_name = __handle_substitute_nouns(new_person, person_words[0])
        
                # wenn keine Namen sondern nur substitute_nouns gefunden,
                #  dann wird neue Person nicht liste hinzugefügt
                if len(person_name) > 0:
                    new_person.first_name = person_name[0]
                    if len(person_name) > 1:
                        # um Nachnamen wie "Le Clos" oder "Von Niederhaeusern" zu erkennen
                        last_name = " ".join(person_name[1:])
                        new_person.last_name = last_name
        
                    new_person.positions_in_text = person_words[1]
        
                    people.append(new_person)
        
            return people        
    \end{lstlisting}
    \caption{Function \_\_get\_people\_from\_ner\_coref}
    \label{get-people-from-ner-coref}
\end{figure}

\begin{figure}[H]
	\begin{lstlisting}[language=Python]
    class Person:
        first_name: str
        last_name: str
        pronouns_and_articles: list[str]  # z.B.: er, sie, ihr, ihre, sein, seine der, die, das
        substitute_nouns: list[str]  # z.B.: Informatikerin, Studentin, Schwester, Tochter, Prinz, Experte
        positions_in_text: list[int]
	\end{lstlisting}
	\caption{Person Klasse}
	\label{person-class}
\end{figure}